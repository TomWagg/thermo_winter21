\documentclass[12pt, letterpaper, twoside]{article}
\usepackage{nopageno,epsfig, amsmath, amssymb}
\usepackage{physics}
\usepackage{mathtools}
\usepackage{hyperref}
\usepackage{xcolor}
\usepackage{mhchem}
\hypersetup{
    colorlinks,
    linkcolor={blue},
    citecolor={blue},
    urlcolor={blue}
}
\usepackage{empheq}
\usepackage{wrapfig}
\usepackage[shortlabels]{enumitem}

\usepackage[letterpaper,
            margin=0.8in]{geometry}

\newcommand{\psetnum}{1}
\newcommand{\class}{ASTR 531 - Stellar Interiors and Evolution}

\newcommand{\tomtitle}{
    \noindent {\LARGE \fontfamily{cmr}\selectfont \textbf{\class}} \hfill \\[1\baselineskip]
    \noindent {\large \fontfamily{cmr}\selectfont Exam \psetnum \hfill \textsc{Tom Wagg}}\\[0.5\baselineskip]
    {\fontfamily{cmr}\selectfont \textit{\today}}\\[2\baselineskip]
}

\title{\class : Problem Set \psetnum}
\author{\textbf{Tom Wagg}}

\newcommand{\question}[1]{{\noindent \it #1}}
\newcommand{\answer}[1]{
    \par\noindent\rule{\textwidth}{0.4pt}#1\vspace{0.5cm}
}
\newcommand{\todo}[1]{{\color{red}\begin{center}TODO: #1\end{center}}}

% custom function for adding units
\makeatletter
\newcommand{\unit}[1]{%
    \,\mathrm{#1}\checknextarg}
\newcommand{\checknextarg}{\@ifnextchar\bgroup{\gobblenextarg}{}}
\newcommand{\gobblenextarg}[1]{\,\mathrm{#1}\@ifnextchar\bgroup{\gobblenextarg}{}}
\makeatother

\newcommand{\avg}[1]{\left\langle #1 \right\rangle}
\newcommand{\angstrom}{\mbox{\normalfont\AA}}
\allowdisplaybreaks

\begin{document}

\tomtitle

\vspace{-1.5cm}

\section*{Thriving under high pressure}
\question{Dearest Grad Student,\\

\noindent My name is Professor Daniel Eyemund and I'm reaching out to you with an exciting business idea - diamonds! I've realised that I could get around applying for grants by generating a large number of diamonds and discretely selling them! They might be rare on Earth but with the right conditions perhaps we could create a \textbf{bunch} of them out in space. The plan is as follows:
\begin{enumerate}[nosep]
    \item Collect a big cloud of carbon
    \item Squish it down and inject it into a white dwarf
    \item Create diamonds
    \item \dots
    \item Profit?
\end{enumerate}
Your mission, should you choose to accept it (\dots which you have to because this is an exam), is to find out whether a white dwarf could produce diamonds, if so, what kind of white dwarf we need! I feel your experiences with stellar interiors, not to mention high pressure environments, would make you ideally suited to this task. Please note your timely response would be appreciated as I will be meeting with Elon to pitch the idea in 30 minutes.\\\\
Yours sincerely,

\noindent Prof. D. Eyemund}
\answer{}

\question{\textbf{Part a - Complete degeneracy pressure for electrons}}

\question{Given that, for complete degeneracy, the $n_e(p)$ distribution is rectangular for $p < p_{\rm F}$ such that $n_e(p) = 2 / h^3$, show that the pressure scales as $P_e \propto n_e^{5/3}$.}

\answer{}

\question{\textbf{Part b - Polytrope central pressure}}

\question{By showing that part a is true, you've proven that we can use an $n = 3/2$ polytrope! It'll now be useful to know an expression for the central pressure of a polytrope star.\\

\noindent Use the fact that $P_c = K \rho_c^{1 + 1/n}$ to show that $P_c = A \frac{G M^2}{R^4}$ (and find the constant $A$ in terms of $D_n, \pi, n, M_n$). Hint: You can find $K$ by comparing an expression for $\alpha$ and an expression for $M_n$. Additionally note that $\rho_c = \bar{\rho} D_n$.}

\answer{
    To start, we need to find an expression for $K$. We have the definition of $\alpha$ from the Lane-Emden equation (see Eq. 11.5).
    \begin{equation}
        \alpha = \qty[\frac{(n + 1) K}{4 \pi G \rho_c^{1 - 1/n}}]^{1/2}
    \end{equation}
    Next we can use Equation 11.10a to find another expression for $\alpha$ from the definition of $M_n$
    \begin{align}
        M &= 4 \pi a^3 \rho_c M_n \\
        \alpha &= \qty(\frac{M}{M_n 4 \pi \rho_c})^{1/3}
    \end{align}
    Combining these yields an expression for $K$
    \begin{align}
        \qty[\frac{(n + 1) K}{4 \pi G \rho_c^{1 - 1/n}}]^{1/2} &= \qty(\frac{M}{M_n 4 \pi \rho_c})^{1/3} \\
        K &= \qty(\frac{M}{M_n 4 \pi \rho_c})^{2/3} \frac{4 \pi G \rho_c^{1 - 1 / n}}{(n + 1)} \\
        K &= \qty(\frac{M}{M_n})^{2/3} \frac{(4 \pi)^{1/3} G \rho_c^{1/3 - 1 / n}}{(n + 1)}
    \end{align}
    Now, given that $P_c = K \rho_c^{1 + 1/n}$, this means that we can write the central pressure as
    \begin{align}
        P_c &= \qty(\frac{M}{M_n})^{2/3} \frac{(4 \pi)^{1/3} G \rho_c^{1/3 - 1 / n}}{(n + 1)} \rho_c^{1 + 1 / n} \\
            &= \qty(\frac{M}{M_n})^{2/3} \frac{(4 \pi)^{1/3} G \rho_c^{4/3}}{(n + 1)} \\
            &= \qty[\frac{(4 \pi)^{1/3} G}{(n + 1) M_n^{2/3}}] M^{2/3} \rho_c^{4/3}
    \end{align}
    We also know that the average density is given by $\bar{\rho} = \frac{M}{\frac{4}{3} \pi R^3}$ and it is related to the central density as $\rho_c = \bar{\rho} D_n$. Therefore we can express the central pressure
    \begin{align}
        P_c &= \qty[\frac{(4 \pi)^{1/3} G}{(n + 1) M_n^{2/3}}] M^{2/3} \qty(\frac{M D_n}{\frac{4}{3} \pi R^3})^{4/3} \\
            &= \qty[\frac{(4 \pi)^{1/3} G}{(n + 1) M_n^{2/3}}] \qty(\frac{D_n}{\frac{4}{3} \pi})^{4/3} \frac{M^2}{R^4} \\
        P_c &= \underbrace{\qty[\frac{(3 D_n)^{4/3}}{4 \pi (n + 1) M_n^{2/3}}]}_{=A} \frac{G M^2}{R^4}
    \end{align}
}

\question{\textbf{Part c - White dwarf $\mathbf{M}$-$\mathbf{P_c}$ relation}}

\question{Now let's take that and create a relation for the central pressure of a white dwarf as only a function of mass. Apply the $M$-$R$ relation for WDs (assuming $\mu_e = 2$) and that $n = 3/2$ to show that $P_c = \frac{A G}{B^4} M^{C}$ where $B$ and $C$ are constants you should find.}

\answer{
    The WD $M$-$R$ relation with $\mu_e = 2$ gives that
    \begin{equation}
        R = 0.012 \unit{R_{\odot}} \qty(\frac{M}{\unit{M_{\odot}}})^{-1/3}
    \end{equation}
    This gives the final result of the central pressure as a function of WD mass
    \begin{equation}
        \boxed{ P_{c, {\rm WD}} = \frac{A G}{B^4} M^{C} }
    \end{equation}
    where $B = 0.012 \unit{R_{\odot}} \unit{M_{\odot}}^{1/3}$ and $C = 10/3$.
}

\end{document}

 