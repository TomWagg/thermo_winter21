\documentclass[12pt, letterpaper, twoside]{article}
\usepackage{nopageno,epsfig, amsmath, amssymb}
\usepackage{physics}
\usepackage{mathtools}
\usepackage{hyperref}
\usepackage{xcolor}
\usepackage{mhchem}
\usepackage[normalem]{ulem}
\hypersetup{
    colorlinks,
    linkcolor={blue},
    citecolor={blue},
    urlcolor={blue}
}
\usepackage{empheq}
\usepackage{wrapfig}
\usepackage[shortlabels]{enumitem}

\usepackage[letterpaper,
            margin=0.8in]{geometry}

\newcommand{\psetnum}{2}
\newcommand{\class}{ASTR 531 - Stellar Interiors and Evolution}

\newcommand{\tomtitle}{
    \noindent {\LARGE \fontfamily{cmr}\selectfont \textbf{\class}} \hfill \\[1\baselineskip]
    \noindent {\large \fontfamily{cmr}\selectfont Exam \psetnum \hfill \textsc{Tom Wagg}}\\[0.5\baselineskip]
    {\fontfamily{cmr}\selectfont \textit{\today}}\\[2\baselineskip]
}

\title{\class : Exam \psetnum}
\author{\textbf{Tom Wagg}}

\newcommand{\question}[1]{{\noindent \it #1}}
\newcommand{\answer}[1]{
    \par\noindent\rule{\textwidth}{0.4pt}#1\vspace{0.5cm}
}
\newcommand{\todo}[1]{{\color{red}\begin{center}TODO: #1\end{center}}}

% custom function for adding units
\makeatletter
\newcommand{\unit}[1]{%
    \,\mathrm{#1}\checknextarg}
\newcommand{\checknextarg}{\@ifnextchar\bgroup{\gobblenextarg}{}}
\newcommand{\gobblenextarg}[1]{\,\mathrm{#1}\@ifnextchar\bgroup{\gobblenextarg}{}}
\makeatother

\newcommand{\avg}[1]{\left\langle #1 \right\rangle}
\newcommand{\angstrom}{\mbox{\normalfont\AA}}
\allowdisplaybreaks

\begin{document}

\tomtitle

\vspace{-1.5cm}

\section*{Getting a white dwarf to chill out}
\vspace{-0.35cm}
\subsubsection*{Prof.\ Eyemund Returns}
\vspace{0.25cm}
\question{Dearest Grad Student,\\

    \noindent I have good news and bad news. The bad news is that it seems that our venture into the diamond creation business has gone slightly awry. The other faculty became suspicious after they noticed I was using that goblet of diamonds as a paperweight and after some lengthy arguments I have been legally forbidden from ever using white dwarfs for diamond production! The good news? Well they didn't say anything about neutron stars...\\
    
    \vspace{-0.3cm}
    \noindent Now the problem of course is that we've invested quite a lot of \sout{your PhD funding} money into finding that sample of white dwarfs. But I realised we don't need to \emph{find} a neutron star, we just need to \emph{make} a neutron star!!\\

    \vspace{-0.3cm}
    \noindent So I've done some experiments with adding mass to existing white dwarfs, but for some reason none of them reach the Chandrasekhar limit before exploding dramatically. I'm not entirely sure why this happens but it seems to be temperature dependent, therefore I've acquired a Cooling Ray to stop the envelope from getting too hot! It can cool at any rate we want, but it costs more to do so. So my question for you is \textbf{What's the lowest rate we could cool the outer layers of a white dwarf whilst it accretes material and still create a neutron star?}\\\\
    Yours sincerely,
    
    \noindent Prof. D. Eyemund
    \answer{}
}

\question{\textbf{Part a - What's with those explosions?}}

\question{Explain what is physically occuring at the surface of the white dwarf that leads to exoplosions during accretion.}

\answer{
    It's the ignition of thermonuclear fusion! As the white dwarf accretes material, the temperature of the envelope starts to heat up. Eventually, the temperature reaches the ignition temperature of H-fusion and thus fusion is rapidly ignited throughout the envelope, leading to an exoplosive ejection of the accreted shell of material.
}

\clearpage

\question{\textbf{Part b - How quickly does an accreting WD heat up?}}

\question{Derive the rate at which an accreting WD heats:
\begin{equation}
    \dot{T} = f(M_{\rm WD}, R_{\rm WD}, M_{\rm env}, \mu, \dot{M}_a)
\end{equation}
It should be a function of the mass of the white dwarf, radius of the white dwarf, mass of the surrounding envelope, mean molecular weight of envelope and the accretion rate.

You should assume that all gravitational energy of the accreted material is converted to thermal energy in the WD envelope during accretion and that the white dwarf is already very close to the Chandrasekhar limit. Additionally, you can assume the envelope is an ideal gas. and its mass remains approximately constant.}

\answer{
    The gravitational energy for the accreted mass on the white dwarf is just a function of the masses and the separation.
    \begin{equation}
        E_{\rm grav} = - \frac{G M_{\rm WD} M_a}{R_{\rm WD}}
    \end{equation}
    Since the envelope of the WD is an ideal gas, the average thermal energy is
    \begin{equation}
        \avg{E_{\rm thermal}} = \frac{3}{2} k_B T
    \end{equation}
    We can convert this to the total thermal energy of the envelope by multiplying by the total number of particles in the envelope
    \begin{equation}
        E_{\rm thermal} = \avg{E_{\rm thermal}} \cdot N = \frac{3}{2} k_B T \cdot \frac{M_{\rm env}}{\mu m_H}
    \end{equation}
    Now, since we are assuming that all gravitational energy is converted to thermal energy then the time derivatives of these expressions should be equal and we can solve for the change in temperature
    \begin{align}
        - \dv{E_{\rm grav}}{t} &= \dv{E_{\rm thermal}}{t} \\
        \dv{t} \frac{G M_{\rm WD} M_a}{R_{\rm WD}} &= \dv{t} \frac{3}{2} k_B T \cdot \frac{M_{\rm env}}{\mu m_H} \\
        \frac{G M_{\rm WD} \dot{M}_a}{R_{\rm WD}} &= \frac{3}{2} k_B \dot{T} \cdot \frac{M_{\rm env}}{\mu m_H} \\
        \Aboxed{ \dot{T} &= \frac{2}{3 k_B} \frac{\mu m_H}{M_{\rm env}} \frac{G M_{\rm WD} \dot{M}_a}{R_{\rm WD}} }
    \end{align}
}

\question{\textbf{Part c - How fast does the laser need to cool the WD?}}

\question{Calculate the minimum cooling rate required for the WD to reach the Chandrasekhar limit before any explosions occur.}

\end{document}

 