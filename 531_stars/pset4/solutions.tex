\documentclass[12pt, letterpaper, twoside]{article}
\usepackage{nopageno,epsfig, amsmath, amssymb}
\usepackage{physics}
\usepackage{mathtools}
\usepackage{hyperref}
\usepackage{xcolor}
\usepackage{mhchem}
\hypersetup{
    colorlinks,
    linkcolor={blue},
    citecolor={blue},
    urlcolor={blue}
}
\usepackage{empheq}
\usepackage{wrapfig}

\usepackage[letterpaper,
            margin=0.8in]{geometry}

\newcommand{\psetnum}{4}
\newcommand{\class}{ASTR 531 - Stellar Interiors and Evolution}

\newcommand{\tomtitle}{
    \noindent {\LARGE \fontfamily{cmr}\selectfont \textbf{\class}} \hfill \\[1\baselineskip]
    \noindent {\large \fontfamily{cmr}\selectfont Problem Set \psetnum \hfill \textsc{Tom Wagg}}\\[0.5\baselineskip]
    {\fontfamily{cmr}\selectfont \textit{\today}}\\[2\baselineskip]
}

\title{\class : Problem Set \psetnum}
\author{\textbf{Tom Wagg}}

\newcommand{\question}[1]{{\noindent \it #1}}
\newcommand{\answer}[1]{
    \par\noindent\rule{\textwidth}{0.4pt}#1\vspace{0.5cm}
}
\newcommand{\todo}[1]{{\color{red}\begin{center}TODO: #1\end{center}}}

% custom function for adding units
\makeatletter
\newcommand{\unit}[1]{%
    \,\mathrm{#1}\checknextarg}
\newcommand{\checknextarg}{\@ifnextchar\bgroup{\gobblenextarg}{}}
\newcommand{\gobblenextarg}[1]{\,\mathrm{#1}\@ifnextchar\bgroup{\gobblenextarg}{}}
\makeatother

\newcommand{\avg}[1]{\left\langle #1 \right\rangle}
\newcommand{\angstrom}{\mbox{\normalfont\AA}}
\allowdisplaybreaks

\begin{document}

\tomtitle

\question{\textbf{20.2 - White Dwarf Luminosity}}

\question{Part a - Difference in luminosity}
\answer{
    The luminosity of a white dwarf in the slow cooling phase is given by Eq.\ 20.10 in the textbook
    \begin{equation}
        \frac{L}{\unit{L_\odot}} \approx 5.2 \times 10^{10} \frac{M}{\unit{M_\odot}} \mu_{\rm ion}^{-7/5} \qty(\frac{t}{\unit{yr}})^{-7.5}
    \end{equation}
    Since we are comparing white dwarfs with the same cooling age, the only relevant factors are the mass and $\mu_{\rm ion}$ when comparing a H-rich WD to He-rich and C-rich.
    
    The values of $\mu_{\rm ion}$ for these WDs are $1, 4, 12$ respectively. This means that the relative luminosity of the WDs is
    \begin{equation}
        \frac{L}{L_{\rm H-rich}} = 1 : 0.14 : 0.03
    \end{equation}
    respectively. This shows that H-rich WDs are the brightest for a given cooling age, following my He-rich and then C-rich WDs.
}

\question{Part b - Reason for differences}
\answer{
    The cooling of a WD is due to the loss of the thermal energy of the ions. The specific heat of a substance gives a measure for how much energy is required to change its temperature by a given amount. Therefore, a smaller specific heat means that cooling an object is easier and will happen more quickly. The specific heat of the ion is given by
    \begin{equation}
        c_V = \frac{3 k}{2 m_{\rm ion}}
    \end{equation}
    This means that for helium and carbon, the increased $m_{\rm ion}$ results in a smaller specific heat and thus faster cooling rate.
}

\question{\textbf{23.2 - Central Temperature-Density Gradient}}
\answer{
    For most of its life the star follows
    \begin{equation}
        T_c \sim \rho_c^{1/3},
    \end{equation}
    however this trend becomes less steep over time. This is because the full expression is actually
    \begin{equation}
        T_c \sim M_c^{2/3} \rho_c^{1/3},
    \end{equation}
    For the later fusion phases, the core mass becomes much lower (see Eq.\ 14.10) and so we can't assume it to be constant. Therefore, we see that for late fusion phases the increase in temperature for a given density is lower, thus the evolution in the plot is less steep.
}

\question{\textbf{25.1 - Non-spherical mass loss in rotating stars}}

\question{Part a - Mass loss latitude trends}
\answer{
    The star in Figure 25.3 is very massive, for these stars the winds are driven mainly by radiation pressure. The Von Zeipel effect means that the effective temperature of the pole is higher than at the equator. This means that for most cases (the left panel) mass loss is much higher at the poles than at the equator.

    In certain cases (the right panel), the decreased $T_{\rm eff}$ at the equator actually leads to a higher mass loss rate due to the formation of a bi-stability disk. This disk has higher mass loss because its lower effective temperature puts it on the other side of a bi-stability jump from the pole and so its winds are driven by different ions that more effectively remove mass from the star.
}

\question{Part b - Terminal wind velocity latitude trends}
\answer{
    In rapidly rotating stars, the effective surface gravity at a given radius becomes a function of latitude, with stronger gravity at the poles than the equator. The escape velocity is a direct function of this surface gravity. We have previously shown that the terminal wind velocity is approximately equal to the escape velocity (see Section 15.2.2 of the textbook). For these reasons the terminal wind velocity is higher at the poles than at the equator.
}

\question{Part c - Density of bi-stability disk}
\answer{
    The density of the gas scales as
    \begin{equation}
        \rho \sim \frac{\dot{M}}{v_{\infty}}
    \end{equation}
    For a rotation-induced bi-stability disk, the mass loss is higher than at the poles at its slightly lower $T_{\rm eff}$ places it just below a bi-stability jump (see Section 15.3). Moreover, the terminal wind velocity ($v_\infty$) is lower on the lower side of a bi-stability jump (see Section 15.2.2). The combination of these effects leads to a significant increase in density in the disk relative to the poles.
}

\question{\textbf{27.1 - Limit for NS vs.\ BH for rotating stars}}
\answer{
    The limit of stars with initial masses resulting in NSs or BHs could change due to rotation in the following ways:
    \begin{itemize}
        \item \textit{Case for higher mass limit:} The mass needed to create a BH would need to increase to compensate for the increased mass loss due to rotation. Otherwise, less mass is available to go into the creation of a remnant and so stars that would have resulted in BHs instead result in NSs.
        \item \textit{Case for lower mass limit:} Though the overall mass is decreased for a rapidly-rotating star, the mass of the helium core \textit{increases} due to the induced mixing. A larger core mass leads to a larger remnant mass and thus more easily create a BH with a given initial mass.
    \end{itemize}
    Overall, the winner here is the latter reasoning. Core mass is the most important parameter for determining the destiny of any massive star and so if all stars were rapidly-rotating the mass limit between NSs and BHs would decrease.
}

\end{document}

 