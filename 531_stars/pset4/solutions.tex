\documentclass[12pt, letterpaper, twoside]{article}
\usepackage{nopageno,epsfig, amsmath, amssymb}
\usepackage{physics}
\usepackage{mathtools}
\usepackage{hyperref}
\usepackage{xcolor}
\usepackage{mhchem}
\hypersetup{
    colorlinks,
    linkcolor={blue},
    citecolor={blue},
    urlcolor={blue}
}
\usepackage{empheq}
\usepackage{wrapfig}

\usepackage[letterpaper,
            margin=0.8in]{geometry}

\newcommand{\psetnum}{4}
\newcommand{\class}{ASTR 531 - Stellar Interiors and Evolution}

\newcommand{\tomtitle}{
    \noindent {\LARGE \fontfamily{cmr}\selectfont \textbf{\class}} \hfill \\[1\baselineskip]
    \noindent {\large \fontfamily{cmr}\selectfont Problem Set \psetnum \hfill \textsc{Tom Wagg}}\\[0.5\baselineskip]
    {\fontfamily{cmr}\selectfont \textit{\today}}\\[2\baselineskip]
}

\title{\class : Problem Set \psetnum}
\author{\textbf{Tom Wagg}}

\newcommand{\question}[1]{{\noindent \it #1}}
\newcommand{\answer}[1]{
    \par\noindent\rule{\textwidth}{0.4pt}#1\vspace{0.5cm}
}
\newcommand{\todo}[1]{{\color{red}\begin{center}TODO: #1\end{center}}}

% custom function for adding units
\makeatletter
\newcommand{\unit}[1]{%
    \,\mathrm{#1}\checknextarg}
\newcommand{\checknextarg}{\@ifnextchar\bgroup{\gobblenextarg}{}}
\newcommand{\gobblenextarg}[1]{\,\mathrm{#1}\@ifnextchar\bgroup{\gobblenextarg}{}}
\makeatother

\newcommand{\avg}[1]{\left\langle #1 \right\rangle}
\newcommand{\angstrom}{\mbox{\normalfont\AA}}
\allowdisplaybreaks

\begin{document}

\tomtitle

\question{\textbf{20.2 - White Dwarf Luminosity}}

\question{Part a}
\answer{
    The luminosity of a white dwarf in the slow cooling phase is given by Eq.\ 20.10 in the textbook
    \begin{equation}
        \frac{L}{\unit{L_\odot}} \approx 5.2 \times 10^{10} \frac{M}{\unit{M_\odot}} \mu_{\rm ion}^{-7/5} \qty(\frac{t}{\unit{yr}})^{-7.5}
    \end{equation}
    Since we are comparing white dwarfs with the same cooling age, the only relevant factors are the mass and $\mu_{\rm ion}$ when comparing a H-rich WD to He-rich and C-rich.
    
    The values of $\mu_{\rm ion}$ for these WDs are $1, 4, 12$ respectively. This means that the relative luminosity of the WDs is
    \begin{equation}
        \frac{L}{L_{\rm H-rich}} = 1 : 0.14 : 0.03
    \end{equation}
    respectively. This shows that H-rich WDs are the brightest for a given cooling age, following my He-rich and then C-rich WDs.

    \todo{Can we assume that the mass is constant?}
}

\question{Part b}
\answer{
    \todo{Unsure, maybe larger ions make cooling happen faster? Why?}
}

\question{\textbf{23.2 - Central Temperature-Density Gradient}}
\answer{}



\end{document}

 