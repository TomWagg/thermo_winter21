\documentclass[12pt, letterpaper, twoside]{article}
\usepackage{nopageno,epsfig, amsmath, amssymb}
\usepackage{physics}
\usepackage{mathtools}
\usepackage{hyperref}
\usepackage{xcolor}
\usepackage{mhchem}
\hypersetup{
    colorlinks,
    linkcolor={blue},
    citecolor={blue},
    urlcolor={blue}
}
\usepackage{empheq}
\usepackage{wrapfig}

\usepackage[letterpaper,
            margin=0.8in]{geometry}

\newcommand{\psetnum}{3}
\newcommand{\class}{ASTR 531 - Stellar Interiors and Evolution}

\newcommand{\tomtitle}{
    \noindent {\LARGE \fontfamily{cmr}\selectfont \textbf{\class}} \hfill \\[1\baselineskip]
    \noindent {\large \fontfamily{cmr}\selectfont Problem Set \psetnum \hfill \textsc{Tom Wagg}}\\[0.5\baselineskip]
    {\fontfamily{cmr}\selectfont \textit{\today}}\\[2\baselineskip]
}

\title{\class : Problem Set \psetnum}
\author{\textbf{Tom Wagg}}

\newcommand{\question}[1]{{\noindent \it #1}}
\newcommand{\answer}[1]{
    \par\noindent\rule{\textwidth}{0.4pt}#1\vspace{0.5cm}
}
\newcommand{\todo}[1]{{\color{red}\begin{center}TODO: #1\end{center}}}

% custom function for adding units
\makeatletter
\newcommand{\unit}[1]{%
    \,\mathrm{#1}\checknextarg}
\newcommand{\checknextarg}{\@ifnextchar\bgroup{\gobblenextarg}{}}
\newcommand{\gobblenextarg}[1]{\,\mathrm{#1}\@ifnextchar\bgroup{\gobblenextarg}{}}
\makeatother

\newcommand{\avg}[1]{\left\langle #1 \right\rangle}
\newcommand{\angstrom}{\mbox{\normalfont\AA}}
\allowdisplaybreaks

\begin{document}

\tomtitle

\question{\textbf{12.2 - Early Radii and Timescales}}

\question{Part a - Radii Estimations}
\answer{
    Let's use a couple of different relations from the textbook to get the radii at different times. A protostar becomes ionised and stars the Hayashi concentration phase when it's radius is on the order of
    \begin{equation}
        R_{\rm Hayashi, start} \approx 100 \unit{R_{\odot}} \qty(\frac{M}{\unit{M_\odot}})
    \end{equation}
    We find the radius of the protostar once the Hayashi concentration phase comes to an end is approximately a factor of 50 lower (based on assumptions of the temperature and opacity) such that
    \begin{equation}
        R_{\rm Hayashi, end} \approx 2 \unit{R_{\odot}} \qty(\frac{M}{\unit{M_\odot}})
    \end{equation}
    The radius at the start of the PMS phase will be the same as the end of the Hayashi concentration phase.
    \begin{equation}
        R_{\rm PMS, start} = R_{\rm Hayashi, end}
    \end{equation}
    Finally, the radius at the end of the PMS phase is the same as the radius at ZAMS and so we can write that
    \begin{equation}
        R_{\rm PMS, end} = R_{\rm ZAMS} = \unit{R_\odot} \qty(\frac{M}{\unit{M_\odot}})^{0.7}
    \end{equation}
    So now we can plug in numbers for the different masses of stars that we considered
    \begin{center}
        \begin{tabular}{c|cccc}
            $M / \unit{M_\odot}$ & $R_{\rm Hayashi, start} / \unit{R_\odot}$ & $R_{\rm Hayashi, end} / \unit{R_\odot}$ & $R_{\rm PMS, start} / \unit{R_\odot}$ & $R_{\rm PMS, end} / \unit{R_\odot}$ \\
            \hline
            0.3 & 30 & 0.6 & 0.6 & 0.43 \\
            3 & 300 & 6 & 6 & 2.16 \\
            30 & 3000 & 60 & 60 & 10.8 \\
        \end{tabular}
    \end{center}
}

\question{\textbf{15.4 - Metallicity and Mass Loss Rates}}
\answer{}

\question{\textbf{16.1 - RGB Radii}}
\answer{}

\question{\textbf{17.1 - Helium Flash Duration}}
\answer{}

\end{document}

 