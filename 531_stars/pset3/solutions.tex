\documentclass[12pt, letterpaper, twoside]{article}
\usepackage{nopageno,epsfig, amsmath, amssymb}
\usepackage{physics}
\usepackage{mathtools}
\usepackage{hyperref}
\usepackage{xcolor}
\usepackage{mhchem}
\hypersetup{
    colorlinks,
    linkcolor={blue},
    citecolor={blue},
    urlcolor={blue}
}
\usepackage{empheq}
\usepackage{wrapfig}

\usepackage[letterpaper,
            margin=0.8in]{geometry}

\newcommand{\psetnum}{3}
\newcommand{\class}{ASTR 531 - Stellar Interiors and Evolution}

\newcommand{\tomtitle}{
    \noindent {\LARGE \fontfamily{cmr}\selectfont \textbf{\class}} \hfill \\[1\baselineskip]
    \noindent {\large \fontfamily{cmr}\selectfont Problem Set \psetnum \hfill \textsc{Tom Wagg}}\\[0.5\baselineskip]
    {\fontfamily{cmr}\selectfont \textit{\today}}\\[2\baselineskip]
}

\title{\class : Problem Set \psetnum}
\author{\textbf{Tom Wagg}}

\newcommand{\question}[1]{{\noindent \it #1}}
\newcommand{\answer}[1]{
    \par\noindent\rule{\textwidth}{0.4pt}#1\vspace{0.5cm}
}
\newcommand{\todo}[1]{{\color{red}\begin{center}TODO: #1\end{center}}}

% custom function for adding units
\makeatletter
\newcommand{\unit}[1]{%
    \,\mathrm{#1}\checknextarg}
\newcommand{\checknextarg}{\@ifnextchar\bgroup{\gobblenextarg}{}}
\newcommand{\gobblenextarg}[1]{\,\mathrm{#1}\@ifnextchar\bgroup{\gobblenextarg}{}}
\makeatother

\newcommand{\avg}[1]{\left\langle #1 \right\rangle}
\newcommand{\angstrom}{\mbox{\normalfont\AA}}
\allowdisplaybreaks

\begin{document}

\tomtitle

\question{\textbf{12.2 - Early Radii and Timescales}}

\question{Part a - Radii Estimations}
\answer{
    Let's use a couple of different relations from the textbook to get the radii at different times. A protostar becomes ionised and stars the Hayashi concentration phase when it's radius is on the order of (Eq. 12.13)
    \begin{equation}
        R_{\rm Hayashi, start} \approx 100 \unit{R_{\odot}} \qty(\frac{M}{\unit{M_\odot}})
    \end{equation}
    We find the radius of the protostar once the Hayashi concentration phase comes to an end is approximately a factor of 50 lower (based on assumptions of the temperature and opacity) such that (page 12-8)
    \begin{equation}
        R_{\rm Hayashi, end} \approx 2 \unit{R_{\odot}} \qty(\frac{M}{\unit{M_\odot}})
    \end{equation}
    The radius at the start of the PMS phase will be the same as the end of the Hayashi concentration phase.
    \begin{equation}
        R_{\rm PMS, start} = R_{\rm Hayashi, end}
    \end{equation}
    Finally, the radius at the end of the PMS phase is the same as the radius at ZAMS and so we can write that (Eq. 12.16)
    \begin{equation}
        R_{\rm PMS, end} = R_{\rm ZAMS} = \unit{R_\odot} \qty(\frac{M}{\unit{M_\odot}})^{0.7}
    \end{equation}
    So now we can plug in numbers for the different masses of stars that we considered
    \begin{center}
        \begin{tabular}{c|cccc}
            $M / \unit{M_\odot}$ & $R_{\rm Hayashi, start} / \unit{R_\odot}$ & $R_{\rm Hayashi, end} / \unit{R_\odot}$ & $R_{\rm PMS, start} / \unit{R_\odot}$ & $R_{\rm PMS, end} / \unit{R_\odot}$ \\
            \hline
            0.3 & 30 & 0.6 & 0.6 & 0.43 \\
            3 & 300 & 6 & 6 & 2.16 \\
            30 & 3000 & 60 & 60 & 10.8 \\
        \end{tabular}
    \end{center}
}

\question{Part b - Timescale estimations}
\answer{
    The duration of the Hayashi concentration phase is given in Eq. 12.15 but we also showed that the timescale scales as $1 / M$ such that
    \begin{equation}
        \tau_{\rm Hayashi} \approx 10^{6} \unit{yr} \qty(\frac{M_{\odot}}{M})
    \end{equation}
    The duration of the PMS phase is given by Eq. 12.17 so we have that
    \begin{equation}
        \tau_{\rm PMS} \approx 6 \times 10^7 \unit{yr} \qty(\frac{M}{\unit{M_\odot}})^{-2.5}
    \end{equation}
    So now we can plug in numbers for the different masses of stars that we considered
    \begin{center}
        \begin{tabular}{c|cc}
            $M / \unit{M_\odot}$ & $\tau_{\rm Hayashi} / \unit{yr}$ & $\tau_{\rm PMS} / \unit{yr}$ \\
            \hline
            0.3 & $3.33 \times 10^6$ & $1.22 \times 10^9$ \\
            3 & $3.33 \times 10^5$ & $3.85 \times 10^6$ \\
            30 & $3.33 \times 10^4$ & $1.22 \times 10^4$ \\
        \end{tabular}
    \end{center}
}

\question{\textbf{15.4 - Metallicity and Mass Loss Rates}}
\answer{
    \todo{Need to check code/method with Emily}
}

\question{\textbf{16.1 - RGB Radii}}
\answer{
    From inspection of Figure 16.1 we can find values for $L$ and $T_{\rm eff}$ at the start and end of the RGB phase. This phase starts at C and ends at F. We can then use the fact that
    \begin{equation}
        R = \sqrt{\frac{L}{4 \pi \sigma T_{\rm eff}^4}}
    \end{equation}
    to get the radii. Since I'm in a mood for tables today, let's make another!
    \begin{center}
        \begin{tabular}{c|ccc}
            Stage & $\log (L / \unit{L_\odot})$ & $\log (T_{\rm eff} / \unit{K})$ & $R / \unit{R_{\odot}}$ \\
            \hline
            C (Start of RGB) & 0.4 & 3.7 & 2.1 \\
            F (End of RGB) & 3.4 & 3.48 & 183.1
        \end{tabular}
    \end{center}
    So the radius is increasing by nearly two orders of magnitude!

    \todo{Ask Emily about metallicity}
}

\clearpage

\question{\textbf{17.1 - Helium Flash Duration}}
\answer{
    First we need to calculate the potential energy of a uniform sphere. We know that gravitational potential is given by
    \begin{equation}
        \Phi = - \frac{G M}{r}
    \end{equation}
    So we just need to integrate this over a series of teeny tiny shells over the core of the star. Note we can assume a uniform density so $\rho(r) = \rho$. Let's do it!
    \begin{align}
        U &= - \int_0^R \frac{G M(r)}{r} \rho(r) \dd[3]{r} \\
          &= - \int_0^R \frac{G (\rho(r) \frac{4}{3} \pi r^3)}{r} \rho(r) 4 \pi r^2 \dd{r} \\
          &= - \frac{16 G \pi^2 \rho^2}{3}\int_0^R r^4 \dd{r} \\
        U &= - \frac{16 G \pi^2 \rho^2 R^5}{15}
    \end{align}
    We can rearrange this to get rid of the radius and just write it as a function of density and mass.
    \begin{align}
        R &= \qty(\frac{3 M}{4 \pi \rho})^{1/3} \\
        U &= - \frac{16 G \pi^2 \rho^2}{15} \qty(\frac{3 M}{4 \pi \rho})^{5/3} \\
        U &= - \frac{G \qty(36 \pi M^5 \rho)^{1/3}}{5}
    \end{align}
    Nice, what a...clean expression?\footnote{To be fair, it looks much better if we did it in terms of $M$ and $R$: $U = -\frac{3}{5} G M^2 / R$}. Now let's use this to find the difference in potential energy for a $0.5 \unit{M_\odot}$ core expanding from a density of $\rho = 10^{6} \unit{g}{cm^{-3}}$ to $\rho = 10^{4} \unit{g}{cm^{-3}}$.
    \begin{equation}
        \Delta U = - \frac{G \qty(36 \pi (0.5 \unit{M_\odot})^5)^{1/3}}{5} \qty[(10^{4} \unit{g}{cm^{-3}})^{1/3} - (10^{6} \unit{g}{cm^{-3}})^{1/3}] = 5 \times 10^{42} \unit{J}
    \end{equation}
    Now we can calculate the duration of the Helium flash given the energy production and efficiency
    \begin{equation}
        \tau_{\rm flash} = \frac{\Delta U}{L \cdot \phi} = \frac{5 \times 10^{49} \unit{J}}{10^{10} \unit{L_\odot} \cdot 0.2} = 2.5 \times 10^{40} \unit{erg}{L_\odot^{-1}}
    \end{equation}
    Now because I'm an ``astronomer'' I can 100\% (totally) understand ergs and $L_\odot$ intuitively, but just in case some readers don't live and breathe cgs in the same way this translates to
    \begin{equation}
        \boxed{ \tau_{\rm flash} = 0.21 \unit{yr} \approx 76 \unit{days} }
    \end{equation}
    Quite the `flash' indeed!
}

\end{document}

 