\documentclass[preprint2]{aastex631}
\received{\today}
\shorttitle{Inside-Out Growth}
\graphicspath{{figures/}}

\usepackage{lipsum}
\usepackage{physics}
\usepackage{multirow}
\usepackage{xspace}
\usepackage{natbib}
\usepackage{fontawesome5}
\usepackage{xcolor}
\usepackage{wrapfig}
\usepackage[figuresright]{rotating}

% remove indents in footnotes 
\usepackage[hang,flushmargin]{footmisc} 

\newcommand{\todo}[1]{{\color{red}{[TODO: #1}]}}
\newcommand{\needcite}{{\color{magenta}{(needs citation)}}}
\newcommand{\placeholder}[1]{{\color{gray} \lipsum[#1]}}

% custom function for adding units
\makeatletter
\newcommand{\unit}[1]{%
    \,\mathrm{#1}\checknextarg}
\newcommand{\checknextarg}{\@ifnextchar\bgroup{\gobblenextarg}{}}
\newcommand{\gobblenextarg}[1]{\,\mathrm{#1}\@ifnextchar\bgroup{\gobblenextarg}{}}
\makeatother

\begin{document}

\title{{\Large Inside-Out Growth of Galaxies}\\\vspace{0.15cm}ASTR 511 Final Project}

% affiliations
\newcommand{\UW}{Department of Astronomy, University of Washington, Seattle, WA, 98195}

\author[0000-0001-6147-5761]{Tom Wagg}
\affiliation{\UW}

\correspondingauthor{Tom Wagg}
\email{tomwagg@uw.edu}

\section{Introduction}
For my project I'm going to investigate the inside-out growth of galaxies, focussing specifically on the Milky Way. This is the concept that star formation in spiral galaxies occurs first in the centre and builds outwards over time. This topic was first mentioned by Jim when we discussed chemical cartography and I decided to delve into the topic a bit more deeply (partially motivated by the fact that I used a model including inside-out growth in my paper \citep{Wagg+2022} and I didn't know what that was at the time).

\section{Background \& Early Simulations}
One of the first papers to propose the idea of inside-out growth was \citet{Larson+1976}. They extended previous simulations of galaxies with the goal of producing both a spheroidal and disc component to the galaxy. Most previous papers on the topic \citep[e.g.][]{Larson+1969, Gott+1973} focussed on reproducing the formation of elliptical galaxies and didn't investigate spiral galaxies with a significant disc component.

\citet{Larson+1976} start their simulations with a protogalaxy that is assumed to be a uniform, uniformly rotating sphere with a fixed total mass. Larson addresses the simplicity of these initial conditions by noting that, despite initial uniformity, collapse is always non-homologous and inhomogeneties still emerge.

They explore 9 different models of increasing complexity, including effects such as varying star formation rates, tidally inhibited star formation, different boundary radii. The final model that they find best fits the observations at the time uses a star formation rate that depends on the ratio of the cloud collision time to the free-fall time. This has the effect of producing a more realistic bulge to disc ratio.

\begin{figure}[htb]
    \centering
    \includegraphics[width=\columnwidth]{larson1976_fig13.png}
    \caption{Figure 13 from \citet{Larson+1976}. This shows the star formation rate density as a function of time for different radii for Model 9 from the paper. Annotations indicate the radius, where for example $3.3+2$ means $3.3 \times 10^2$.}
    \label{fig:larson76}
\end{figure}

We show their modelled star formation rate density as a function of time and radius in Figure~\ref{fig:larson76}. They find that there are two distinct phases of star formation: an early burst that is centrally concentrated in the spheroidal component, and a later phase that occurs as gas settles into the plane of the disc. In particular, they note that this phase consists of ``an outward-progressing wave of star formation'' \citep{Larson+1976}. This is one of the first simulations showing a sort of ``inside-out'' growth.

They conclude that high velocity collisions between gas clouds may form spheroidal systems, whilst disc systems may form from more quiescent gas that doesn't require similar collisions. This gas settles onto the disc progressively over time, starting in the centre in which the potential is strongest and then progressing outwards.

\section{Observations}
\subsection{Evidence of inside-out growth}
Using data from 3D-HST and CANDELS Treasury surveys, \citet{vanDokkum+2013} investigated the assembly of Milky-Way-like galaxies since redshift 2.5.

They use the number density of galaxies to rank them and assume that galaxies maintain the same rank throughout cosmic time. This has been shown to be a reasonably effective way of associating progenitor galaxies with their descendants \citep{Leja+2013}. They select galaxies with stellar masses close to that of the Milky Way and use this ranking method to associate galaxies together in order to measure their evolution over time. This produces a sample of approximately 400 galaxies.

After grouping galaxies into bins of redshit, \citet{vanDokkum+2013} stack galaxies and fit them using a \citet{Sersic+1968} profile (correcting for PSF effects and bootstrapping to produce error bars). We show their estimated effective radii for galaxies of different redshifts in Figure~\ref{fig:vd}.

\begin{figure}[htb]
    \centering
    \includegraphics[width=\columnwidth]{vandokkum2013_fig5.png}
    \caption{Figure 5 from \citet{vanDokkum+2013}. The effective radius of observed galaxies from 3D-HST and CANDELS at different redshifts, compared to another paper.}
    \label{fig:vd}
\end{figure}

Figure~\ref{fig:vd} shows that the effective radius of a galaxies decreases with increasing redshift, implying that galaxies are growing over time and thus providing evidence for some sort of inside-out growth. It is also notable that within this figure they also compare to \citet{Patel+2013}, a paper that finds stronger evidence of inside-out growth in more massive galaxies. They conclude that there is likely some sort of mass dependence to the rate of inside-out growth and that galaxy formation models will need to explain both effects.

\subsection{A lack of strong evidence}

\begin{figure*}[t]
    \centering
    \includegraphics[width=\textwidth]{goddard2017_fig14.png}
    \caption{Figure 14 from \citet{Goddard+2017}. Mass weighted age-radius gradients are given in red for four different mass bins, from observations in SDSS-IV MaNGA.}
    \label{fig:goddard}
\end{figure*}

More recently, \citet{Goddard+2017} studied the star formation of histories in galaxies as a function of galaxy mass and type using the SDSS-IV MaNGA survey.

They selected an initial sample of 806 galaxies from the SDSS DR13 release based on whether they were part of Primary sample of MaNGA and absolute magnitude. They use \texttt{FIREFLY} to perform full spectral fitting and cross match with the Galaxy Zoo to establish the morphological classification of the galaxies. Only galaxies with an 80\% majority vote for a certain morphological type were included in the final sample to avoid irregularities. Finally, galaxies with poor velocity dispersion estimates were exluded since they resulted in poor spectral fits. This left a sample of 791 galaxies, of which 216 are spiral galaxies.

\citet{Goddard+2017} use a method of full spectral fitting to spatially resolve the age, metallicity and dust attenuation radial gradients for individual galaxies with the code \texttt{FIREFLY}. They argue that their code improves upon existing codes such as \texttt{STARLIGHT} \citep{CidFernandes+2005}, due to parameter free estimation of dust which requires no dust reddening law and improved underlying stellar tracks. They also check that beam smearing is not a significant effect through comparison with the secondary sample when restricting it to equivalent effective radii. They find that the effect of beam smearing must be quite small and does not affect the gradients in any significant way.

In Figure~\ref{fig:goddard}, we show the age-radius gradients found by \citet{Goddard+2017}. Whilst the light-weighted gradients show a very weak indication of a negative slop, the mass-weighted gradients are not statistically significant and are consistent with zero \citep[in agreement with][]{Sanchez-Blazquez+2014}. Therefore, though the light-weighted gradients show some very slight evidence for inside-out growth, the mass-weighted gradients display no such evidence.

\subsection{Why the disagreement?}
So clearly there is some tension between the results of \citet{vanDokkum+2013} and \citet{Goddard+2017}. Both examine similar samples of galaxies and yet find very different results, indicating that either (1) one of the papers has an error in their analysis or (2) there is another effect at play which is making a difference between the two samples.

With that in mind, we can now consider some recent models for inside-out growth that include a provision for radial migration as a potential solution for this tension.

\section{A model for inside-out growth}
\citet{Frankel+2019} develop and apply a framework for a global evolution Milky Way disc model with an emphasis on inside-out growth. They note the observational tension and highlight that the importance of dynamical heating from spiral arms and molecular clouds has long been recognised. They point out that observations are restricted to studying present-day radii of stellar populations rather than their unknown birth sites and this has a strong effect on the observed gradients.

They argue that, given that we have access to positions, chemical compositions and ages of \textit{individual} stars in the Milky Way, we may be able to trace their birth positions through weak chemical tagging and fit a model for inside-out growth and radial migration simulataneously.

With this goal in mind, they attain a sample of around 5000 red clump giant stars from APOGEE that have asteroseismically calibrated ages, metallicities and 3D positions. These stars in particular were chosen since they have similar core mass and therefore similar luminosities, which makes them useful standard candles. They also focus only on the low-$\alpha$ disc to avoid undesired information from older stars. This means the model mainly describes the recent evolution of the outer disc.

\citet{Frankel+2019} model a variety of effects and have distributions for star formation rates, surface densities and metallicities. We focus in particular in their models for inside-out growth and radial migration. They state that the star formation history is a function of age $\tau$ as
\begin{equation}
    \text{SFH} \propto \exp \qty[ \frac{1}{\tau_{\rm sfr}} \qty( \qty( 1 - x \frac{R_0}{8 \unit{kpc}}) \tau - \tau_m ) ],
\end{equation}
where the inside-out growth is encoded in the dimensionless parameter $x$ ($\tau_{\rm sfr}, R_0, \tau_m$ are other parameters of the model that are fit for). From this equation we can see that a value of $x > 0$ would imply inside-out growth, whilst $x < 0$ is outside-in and $x = 0$ means constant formation across the disc.

\begin{figure}[htb]
    \centering
    \includegraphics[width=\columnwidth]{frankel2019_fig5.png}
    \caption{\citet{Frankel+2019} Figure 5 showing the model for inside-out growth.}
    \label{fig:frankel_io_model}
\end{figure}

This equation is plotting in Figure~\ref{fig:frankel_io_model} for an example set of parameters which demonstrate inside-out growth. One can see that formation for small radii peaks at small radii, whereas formation as large radii is still increasing at present day.

\begin{figure}[htb]
    \centering
    \includegraphics[width=\columnwidth]{frankel2019_fig9.png}
    \caption{}
\end{figure}

\begin{figure}[htb]
    \centering
    \includegraphics[width=\columnwidth]{frankel2019_fig12.png}
    \caption{}
\end{figure}

\section{Applications}
\citep{Banerjee+2020}

\section{Future Work}
\citep{Hogg+2019}

\begin{acknowledgements}
    I thank my laptop for not losing any data despite the fan malfunctioning and the whole thing nearly exploding and I thank GitHub for having my back on that just in case.
    
    Since this is acknowledgements, I will also acknowledge that this isn't my very best work, but it's been quite the quarter and I would like to sleep now and lay this to rest :D
\end{acknowledgements}

\bibliographystyle{aasjournal}
\bibliography{refs}{}

\end{document}