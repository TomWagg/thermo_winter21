\documentclass[12pt, letterpaper, twoside]{article}
\usepackage{nopageno,epsfig, amsmath, amssymb}
\usepackage{physics}
\usepackage{mathtools}
\usepackage{hyperref}
\usepackage{xcolor}
\usepackage{mhchem}
\hypersetup{
    colorlinks,
    linkcolor={blue},
    citecolor={blue},
    urlcolor={blue}
}
\usepackage{empheq}
\usepackage{wrapfig}

\usepackage[letterpaper,
            margin=0.8in]{geometry}

\newcommand{\psetnum}{4}
\newcommand{\class}{ASTR 541 - Interstellar Medium}

\newcommand{\tomtitle}{
    \noindent {\LARGE \fontfamily{cmr}\selectfont \textbf{\class}} \hfill \\[1\baselineskip]
    \noindent {\large \fontfamily{cmr}\selectfont Problem Set \psetnum \hfill \textsc{Tom Wagg}}\\[0.5\baselineskip]
    {\fontfamily{cmr}\selectfont \textit{\today}}\\[2\baselineskip]
}

\title{\class : Week \psetnum}
\author{\textbf{Tom Wagg}}

\newcommand{\question}[1]{{\noindent \it #1}}
\newcommand{\answer}[1]{
    \par\noindent\rule{\textwidth}{0.4pt}#1\vspace{0.5cm}
}
\newcommand{\todo}[1]{{\color{red}\begin{center}TODO: #1\end{center}}}

% custom function for adding units
\makeatletter
\newcommand{\unit}[1]{%
    \,\mathrm{#1}\checknextarg}
\newcommand{\checknextarg}{\@ifnextchar\bgroup{\gobblenextarg}{}}
\newcommand{\gobblenextarg}[1]{\,\mathrm{#1}\@ifnextchar\bgroup{\gobblenextarg}{}}
\makeatother

\newcommand{\avg}[1]{\left\langle #1 \right\rangle}
\newcommand{\angstrom}{\mbox{\normalfont\AA}}
\allowdisplaybreaks

\begin{document}

\tomtitle{}

\noindent For reference, if you'd ever like to see the code that I've used to get my answers to these, \href{https://github.com/TomWagg/uw-grad-classes/tree/main/541_ism}{here's a link to my GitHub repo}! (\#astropy.units for life)\\

\question{\textbf{1. Collisions}}

\question{Part I: Diagram}

\answer{
    Here's a diagram of a collisional excitation (complete with a Jess level pun)
    \begin{center}
        \includegraphics[width=\textwidth]{~/Downloads/diagram for ISM(1).png}
    \end{center}
}

\question{Part II: Energy Conservation}
\answer{
    We can write down the energy conservation by considering the kinetic energy of the system and the internal energy of species A.
    \begin{equation}
        \boxed{ \frac{1}{2} \mu v_1^2 = h \nu_{21} + \frac{1}{2} \mu v_2^2 }
    \end{equation}
    where $\mu$ is the reduced mass of the two species.
}

\question{Part III: Steady-State Equation}
\answer{
    Assuming a steady-state means that the rate of excitations must equal the rate of \textit{de}-excitations. We can write this in terms of number densities and collisional rate coefficients as
    \begin{equation}
        \boxed{ n_A n_B \gamma_{12} = n_{A*} n_B \gamma_{21} }
    \end{equation}
    where $n_B$ is the number density of species B and $n_A$, $n_{A*}$ are the number densities of species A in the ground and excited state respectively. Each $\gamma$ is the collisional rate coefficient for an excitation or de-excitation respectively.
}

\question{Part IV: Cross Sections}
\answer{
    Now's it time to expand things out a little. We know from lectures that the collisional rate coefficient can be written as
    \begin{equation}
        \gamma_{xy} = \int f(v_x) \sigma_{xy}(v_x) v_x \dd{v_x}
    \end{equation}
    Using this definition, we can transform our steady-state equation into the following
    \begin{equation}
        \boxed{ n_A n_B \int f(v_1) \sigma_{12}(v_1) v_1 \dd{v_1} = n_{A*} n_B \int f(v_2) \sigma_{21}(v_2) v_2 \dd{v_2} }
    \end{equation}
}

\question{Part V: Detailed balance (this is where the magic happens)}
\answer{
    Due to the elegance of detailed balance we can know make all of our lives rather simpler. Detailed balance tells us that we don't need to consider \emph{every} velocity, we can just consider only a single velocity that would excite species A, $v_1$ and resulting velocity from that collision, $v_2$ (and the rest will all cancel out). This means that we can then remove the integral and be left with a rather more pleasant relation.
    \begin{equation}
        \boxed{ n_A n_B f(v_1) \sigma_{12}(v_1) v_1 \dd{v_1} = n_{A*} n_B f(v_2) \sigma_{21}(v_2) v_2 \dd{v_2} }
    \end{equation}
}

\question{Part VI: The Algebra Bit$^{\rm TM}$}
\answer{
    Okie dokie folks, strap in, here we go! First, let's cancel out the $n_B$ (note we could have done this earlier, I didn't do that until now because I wanted to, totally not because I didn't notice).
    \begin{equation}
        n_A f(v_1) \sigma_{12}(v_1) v_1 \dd{v_1} = n_{A*} f(v_2) \sigma_{21}(v_2) v_2 \dd{v_2}
    \end{equation}
    Next it would be good to have a relation between $\dd{v_1}$ and $\dd{v_2}$. Let's go back to our conservation of energy equation (our old friend from part II). Taking the total derivative of each side and setting them equal (since we are in a steady-state), we find that
    \begin{equation}
        v_1 \dd{v_1} = v_2 \dd{v_2}
    \end{equation}
    where the $\mu$ has cancelled. Well that's convenient, now we can just cross out those terms on each side of our equation to get
    \begin{equation}
        f(v_1) \sigma_{12}(v_1) = \frac{n_{A*}}{n_A} f(v_2) \sigma_{21}(v_2)
    \end{equation}
    where I've moved the $n_A$ over to the other side for reasons that will soon become clear (or perhaps just to give myself an air of mystery, take your pick).
    
    \noindent Next let's go through the rather arduous task of expanding the maxwellian terms. Note here that since $v_1$ and $v_2$ are \emph{relative} velocities, so the mass terms are the reduced mass $\mu$ in both cases. This means that the form of the maxwellian we shall looks like this
    \begin{equation}
        f(v) = 4 \pi \qty(\frac{\mu}{2 \pi k_B T})^{\frac{3}{2}} \exp\qty(-\frac{\mu v^2}{2 k_B T}) v^2
    \end{equation}
    Let's plug that into both sides and cancel down
    \begin{align}
        \sigma_{12}(v_1) \qty[4 \pi \qty(\frac{\mu}{2 \pi k_B T})^{\frac{3}{2}} \exp\qty(-\frac{\mu v^2}{2 k_B T}) v^2] &= \frac{n_{A*}}{n_A} \sigma_{21}(v_2) \qty[4 \pi \qty(\frac{\mu}{2 \pi k_B T})^{\frac{3}{2}} \exp\qty(-\frac{\mu v^2}{2 k_B T}) v^2] \\
        \sigma_{12}(v_1) \qty[\exp\qty(-\frac{\mu v_1^2}{2 k_B T}) v_1^2] &= \frac{n_{A*}}{n_A} \sigma_{21}(v_2) \qty[\exp\qty(-\frac{\mu v_2^2}{2 k_B T}) v_2^2] \\
        \sigma_{12}(v_1) v_1^2 &= \frac{n_{A*}}{n_A} \sigma_{21}(v_2) \qty[\exp\qty(-\frac{\mu (v_2^2 - v_1^2)}{2 k_B T}) v_2^2]
    \end{align}
    Jolly good! Now for my big reveal of why I moved the $n_A$ over. Well, since collisions are occurring and changing the excitation levels of atoms in a steady-state we must have an equilibrium between kinetic temperature and excitation temperature. Stretching our memories back to the LTE lecture we can remember that \textbf{this means the Boltzmann equation holds}, which gives
    \begin{equation}
        \frac{n_{A*}}{n_A} = \frac{g_2}{g_1} \exp \qty(- \frac{h\nu}{k_B T})
    \end{equation}
    Splendid! Now let's plug \emph{that} into our equation from before to find
    \begin{align}
        \sigma_{12}(v_1) v_1^2 &= \sigma_{21}(v_2) \frac{g_2}{g_1} \exp\qty(-\frac{h \nu}{k_B T}) \qty[\exp\qty(-\frac{\mu (v_2^2 - v_1^2)}{2 k_B T}) v_2^2] \\
        g_1 \sigma_{12}(v_1) v_1^2 &= \sigma_{21}(v_2) g_2 \qty[\exp\qty(-\frac{\mu (v_2^2 - v_1^2 + \frac{2}{\mu} h \nu)}{2 k_B T}) v_2^2]
    \end{align}
    Now get this, that term in the exponential $(v_2^2 - v_1^2 + \frac{2}{\mu} h \nu)$ is just 0 (from our conservation of energy equation from earlier!). That means we can scrap the exponential and leave ourselves with a tidy little relation between the degeneracies, cross sections and velocities.
    \begin{equation}
        \boxed{ g_1 \sigma_{12}(v_1) v_1^2 = \sigma_{21}(v_2) g_2 v_2^2 }
    \end{equation}
    We made it! Now don't forget to submit your offerings to Detailed Balance to ensure future expressions can be simplified in the same way.
}

\end{document}

 