\documentclass[12pt, letterpaper, twoside]{article}
\usepackage{nopageno,epsfig, amsmath, amssymb}
\usepackage{physics}
\usepackage{mathtools}
\usepackage{hyperref}
\usepackage{xcolor}
\usepackage{array}
\hypersetup{
    colorlinks,
    linkcolor={blue},
    citecolor={blue},
    urlcolor={blue}
}
\usepackage{empheq}
\usepackage{wrapfig}

\usepackage[letterpaper,
            margin=0.8in]{geometry}

\newcommand{\psetnum}{6}
\newcommand{\class}{ASTR 541 - Interstellar Medium}

\newcommand{\tomtitle}{
    \noindent {\LARGE \fontfamily{cmr}\selectfont \textbf{\class}} \hfill \\[1\baselineskip]
    \noindent {\large \fontfamily{cmr}\selectfont Problem Set \psetnum \hfill \textsc{Tom Wagg}}\\[0.5\baselineskip]
    {\fontfamily{cmr}\selectfont \textit{\today}}\\[2\baselineskip]
}

\title{\class : Week \psetnum}
\author{\textbf{Tom Wagg}}

\newcommand{\question}[1]{{\noindent \it #1}}
\newcommand{\answer}[1]{
    \par\noindent\rule{\textwidth}{0.4pt}#1\vspace{0.5cm}
}
\newcommand{\todo}[1]{{\color{red}\begin{center}TODO: #1\end{center}}}

% custom function for adding units
\makeatletter
\newcommand{\unit}[1]{%
    \,\mathrm{#1}\checknextarg}
\newcommand{\checknextarg}{\@ifnextchar\bgroup{\gobblenextarg}{}}
\newcommand{\gobblenextarg}[1]{\,\mathrm{#1}\@ifnextchar\bgroup{\gobblenextarg}{}}
\makeatother

\newcommand{\avg}[1]{\left\langle #1 \right\rangle}
\newcommand{\angstrom}{\mbox{\normalfont\AA}}
\allowdisplaybreaks

\newcolumntype{C}{>{$}c<{$}}

\begin{document}

\tomtitle{}

\noindent For reference, if you'd ever like to see the code that I've used to get my answers to these, \href{https://github.com/TomWagg/uw-grad-classes/tree/main/541_ism}{here's a link to my GitHub repo}! (\#astropy.units for life)\\


\question{\textbf{1. Spectroscopy}}
\answer{
    I'm going to summarise my results in a single table to make it a bit easier to parse.
    \begin{center}
        \boxed{
        \begin{tabular}{l|c|c|c|c|c|c}
            Galaxy & Redshift & SII Ratio & O3HB & N2 & O3N2 & Oxygen Abundance \\
            \hline
            J0929+4644\_172\_157 & 0.017 & 1.27 & 5.45 & 0.05 & 2.06 & 8.17 \\
            J0943+0531\_106\_34 & 0.228 & 1.30 & 0.43 & 0.55 & -0.11 & 8.77
        \end{tabular}}
    \end{center}
    \textbf{Density regimes:} Both of these galaxies are in a low density regime since the SII ratios are both close to 1.44 (which we calculated in the last homework).

    \noindent \textbf{Spectrum Noise:} The second galaxy is noisier because it is at a higher redshift and therefore all fluxes are lower.

    \noindent \textbf{Galaxy Masses:} The second galaxy has a higher oxygen abundance. Therefore I think this means that \textbf{the second galaxy must be more massive}. The reason for this is that massive stars enrich the gas and increase the oxygen abundance. So a higher stellar mass would result in a larger oxygen abundance.
}

\end{document}

 