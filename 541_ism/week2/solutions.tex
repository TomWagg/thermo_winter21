\documentclass[12pt, letterpaper, twoside]{article}
\usepackage{nopageno,epsfig, amsmath, amssymb}
\usepackage{physics}
\usepackage{mathtools}
\usepackage{hyperref}
\usepackage{xcolor}
\usepackage{mhchem}
\hypersetup{
    colorlinks,
    linkcolor={blue},
    citecolor={blue},
    urlcolor={blue}
}
\usepackage{empheq}
\usepackage{wrapfig}

\usepackage[letterpaper,
            margin=0.8in]{geometry}

\newcommand{\psetnum}{2}
\newcommand{\class}{ASTR 541 - Interstellar Medium}

\newcommand{\tomtitle}{
    \noindent {\LARGE \fontfamily{cmr}\selectfont \textbf{\class}} \hfill \\[1\baselineskip]
    \noindent {\large \fontfamily{cmr}\selectfont Problem Set \psetnum \hfill \textsc{Tom Wagg}}\\[0.5\baselineskip]
    {\fontfamily{cmr}\selectfont \textit{\today}}\\[2\baselineskip]
}

\title{\class : Week \psetnum}
\author{\textbf{Tom Wagg}}

\newcommand{\question}[1]{{\noindent \it #1}}
\newcommand{\answer}[1]{
    \par\noindent\rule{\textwidth}{0.4pt}#1\vspace{0.5cm}
}
\newcommand{\todo}[1]{{\color{red}\begin{center}TODO: #1\end{center}}}

% custom function for adding units
\makeatletter
\newcommand{\unit}[1]{%
    \,\mathrm{#1}\checknextarg}
\newcommand{\checknextarg}{\@ifnextchar\bgroup{\gobblenextarg}{}}
\newcommand{\gobblenextarg}[1]{\,\mathrm{#1}\@ifnextchar\bgroup{\gobblenextarg}{}}
\makeatother

\newcommand{\avg}[1]{\left\langle #1 \right\rangle}
\newcommand{\angstrom}{\mbox{\normalfont\AA}}
\allowdisplaybreaks

\begin{document}

\tomtitle{}

\noindent For reference, if you'd ever like to see the code that I've used to get my answers to these, \href{https://github.com/TomWagg/uw-grad-classes/tree/main/541_ism}{here's a link to my GitHub repo}! (\#astropy.units for life)\\

\question{\textbf{1. Maxwellians}}

\answer{
    The minimum velocity needed to ionise a hydrogen atom can be calculated by setting the kinetic energy equal to the ionisation energy
    \begin{align}
        \frac{1}{2} m_{e} v_{\rm ionise}^2 &= E_{\rm ionise} \\
        v_{\rm ionise} &= \sqrt{\frac{2 E_{\rm ionise}}{m_e}} \\
        \Aboxed{ v_{\rm ionise} &= 2187 \unit{km}{s^{-1}}}
    \end{align}
    We now need to integrate the Maxwellian (yay?). Now I don't intend to do that analytically because I am not sadistic and therefore will be numerically integrating it will \texttt{scipy}'s quadrature function. Additionally, it doesn't play nicely with infinity so we're going to be sneaky and flip it around
    \begin{align}
        f_{\rm ionise} &= \int_{v_{\rm ionise}}^{\infty} 4 \pi \qty(\frac{m_e}{2 \pi k_B T})^{\frac{3}{2}} \exp\qty(-\frac{m v^2}{2 k_B T}) v^2 \dd{v} \\
        f_{\rm ionise} &= 1 - \int_0^{v_{\rm ionise}} 4 \pi \qty(\frac{m_e}{2 \pi k_B T})^{\frac{3}{2}} \exp\qty(-\frac{m v^2}{2 k_B T}) v^2 \dd{v}
    \end{align}
    Now we can evaluate this for both temperatures given
    \begin{equation}
        \boxed{ f_{\rm ionise, HII} = 6.5 \times 10^{-7},\quad f_{\rm ionise, halo} = 0.96 }
    \end{equation}
    From this we can conclude that collisional ionisation is extremely unlikely to occur in an HII region, but it will be dominant for hot halo gas.
}

\question{\textbf{2. The Dust Collector}}

\answer{
    The rate at which we collect dust grains will be given by the usual form of
    \begin{equation}
        \mathcal{R} = n \sigma v
    \end{equation}
    We are given a collecting area, $A$, which corresponds to the cross section and velocity, $v_{\rm W}$, so those parts are easy. The more complicated part is that we need to know the number density of the ISM dust particles.\\

    \noindent Luckily for us, the problem from last week is going to be rather relevant! So we can apply what we did in the first parts of last week's problem set
    \begin{align}
        \rho_{\rm ISM} &= 1.4 m_p \cdot n_{\rm H} \\
        \rho_{\rm dust} &= 0.005 \rho_{\rm ISM} \\
        m_{\rm dust, particle} &= \frac{4}{3} \pi a^3 \cdot \rho_{\rm dust, particle} \\
    \end{align}
    Then we can use these things to get the number density of dust
    \begin{align}
        n_{\rm dust} &= \frac{\rho_{\rm dust}}{m_{\rm dust, particle}} \\
                     &= \frac{0.005 \cdot 1.4 m_p \cdot n_{\rm H}}{\frac{4}{3} \pi a^3 \cdot \rho_{\rm dust, particle}} \\
        n_{\rm dust} &= 3.1 \times 10^{-13} \unit{cm^{-3}}
    \end{align}
    Now we can use this to find the required area to get a rate of 1 particle per hour
    \begin{equation}
        \boxed{ A = \frac{\mathcal{R}}{n_{\rm dust} v_W} = \frac{1 \unit{hr^{-1}}}{3 \times 10^{-13} \unit{cm^{-3}} \cdot 26 \unit{km}{s^{-1}}} = 350 \unit{cm^2} }
    \end{equation}

}

\end{document}

 