\documentclass[12pt, letterpaper, twoside]{article}
\usepackage{nopageno,epsfig, amsmath, amssymb}
\usepackage{physics}
\usepackage{mathtools}
\usepackage{hyperref}
\usepackage{xcolor}
\hypersetup{
    colorlinks,
    linkcolor={blue},
    citecolor={blue},
    urlcolor={blue}
}
\usepackage{empheq}
\usepackage{wrapfig}

\usepackage[letterpaper,
            margin=0.8in]{geometry}

\newcommand{\class}{ASTR 541 - Interstellar Medium}

\newcommand{\tomtitle}{
    \noindent {\LARGE \fontfamily{cmr}\selectfont \textbf{\class}} \hfill \\[1\baselineskip]
    \noindent {\large \fontfamily{cmr}\selectfont Final Project Proposal \hfill \textsc{Tom Wagg}}\\[0.5\baselineskip]
    {\fontfamily{cmr}\selectfont \textit{\today}}\\[2\baselineskip]
}

\title{\class : Final Project Proposal}
\author{\textbf{Tom Wagg}}

\newcommand{\question}[1]{{\noindent \it #1}}
\newcommand{\answer}[1]{
    \par\noindent\rule{\textwidth}{0.4pt}#1\vspace{0.5cm}
}
\newcommand{\todo}[1]{{\color{red}\begin{center}TODO: #1\end{center}}}

% custom function for adding units
\makeatletter
\newcommand{\unit}[1]{%
    \,\mathrm{#1}\checknextarg}
\newcommand{\checknextarg}{\@ifnextchar\bgroup{\gobblenextarg}{}}
\newcommand{\gobblenextarg}[1]{\,\mathrm{#1}\@ifnextchar\bgroup{\gobblenextarg}{}}
\makeatother

\newcommand{\avg}[1]{\left\langle #1 \right\rangle}
\newcommand{\angstrom}{\mbox{\normalfont\AA}}
\allowdisplaybreaks

\begin{document}

\tomtitle{}

\vspace{-1.5cm}

\section{General Idea}
I enjoyed the lectures on electronic configuration and spectroscopic terms and I was intrigued by the concept of implementing an algorithm to solve the problems that we did in class by hand. Therefore, for my project I intend to write some code to work out electronic configurations, spectroscopic terms and also create energy level diagrams.

\section{Intended Products}
\subsection{Code}
I'd like to design a small bit of code to do the following. Note in several places I have to say \emph{``(only guaranteed to work for atoms)''} because it seems that Hund's rules and the Aufbau Principle do not apply to ions in the same way - there don't seem to be rules for ions as far as I can tell.\\

\noindent \textbf{Electronic Configurations}
\begin{itemize}
    \item \textbf{Input:} either a number of electrons and ionisations or a string like ``HII'' or ``O3+''
    \item \textbf{Output:} the electronic configuration in a list of $[n, l, n_e)$ tuples - \emph{(only guaranteed to work for atoms)}
    \item Add options to return in ASCII or LaTeX format
\end{itemize}

\noindent \textbf{Spectroscopic Terms}
\begin{itemize}
    \item \textbf{Input:} Subshell parameters, $n, l$ and number of electrons for the shell, $n_e$
    \item \textbf{Output:} the spectroscopic terms in the a list of $(2S+1, L, J)$ tuples - \emph{(only guaranteed to work for atoms)}
    \item Add options to return in ASCII or LaTeX format
    \item Sort terms by energy based on Hund's rules
    \item Work out the ground state spectroscopic terms given an electronic configuration
\end{itemize}

\noindent \textbf{Energy Level diagrams}
\begin{itemize}
    \item \textbf{Input:} a list of spectroscopic terms (or a string to be parsed and it'll calculate them) for a series of levels, a list of transitions between levels as tuples $(n_up, n_low, \lambda)$ - these transitions could be pulled directly from \texttt{linetools}
    \item \textbf{Output:} an energy level diagram with levels labelled with their terms, colour coded and annotated with transition wavelengths
    \item I'm also interesting in adding conditions for whether a transition is forbidden and changing the style of the arrow if it is
    \item So big picture I could get to a point where I write something like \texttt{plot\_energy\_levels("SII", transitions)} and it'll work out the electronic configuration, then the spectroscopic terms, plot up the levels and then add the transitions that you specify.
\end{itemize}

\subsection{Discussion}

I also think I could write/talk about a couple of different things that I found whilst reading:
\begin{itemize}
    \item My implementation of the algorithms for working out the configurations/terms \emph{(writing only, no one wants to hear this in the presentation haha)}
    \item There is a reason for the periodic table's weird shape, we can talk about it!
    \item Symmetry of spectroscopic terms within subshells
    \item The problems with the Aufbau Principle and how electron shielding plays a role in messing it up
\end{itemize}

\section{Format}
My intended format will be a link to the GitHub repository containing the code (which will be fully comments and docstringed), as well as a Jupyter Notebook that demonstrates its use and discusses the points above.


\end{document}

 