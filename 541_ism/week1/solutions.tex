\documentclass[12pt, letterpaper, twoside]{article}
\usepackage{nopageno,epsfig, amsmath, amssymb}
\usepackage{physics}
\usepackage{mathtools}
\usepackage{hyperref}
\usepackage{xcolor}
\hypersetup{
    colorlinks,
    linkcolor={blue},
    citecolor={blue},
    urlcolor={blue}
}
\usepackage{empheq}
\usepackage{wrapfig}

\usepackage[letterpaper,
            margin=0.8in]{geometry}

\newcommand{\psetnum}{1}
\newcommand{\class}{ASTR 541 - Interstellar Medium}

\newcommand{\tomtitle}{
    \noindent {\LARGE \fontfamily{cmr}\selectfont \textbf{\class}} \hfill \\[1\baselineskip]
    \noindent {\large \fontfamily{cmr}\selectfont Problem Set \psetnum \hfill \textsc{Tom Wagg}}\\[0.5\baselineskip]
    {\fontfamily{cmr}\selectfont \textit{\today}}\\[2\baselineskip]
}

\title{\class : Week \psetnum}
\author{\textbf{Tom Wagg}}

\newcommand{\question}[1]{{\noindent \it #1}}
\newcommand{\answer}[1]{
    \par\noindent\rule{\textwidth}{0.4pt}#1\vspace{0.5cm}
}
\newcommand{\todo}[1]{{\color{red}\begin{center}TODO: #1\end{center}}}

% custom function for adding units
\makeatletter
\newcommand{\unit}[1]{%
    \,\mathrm{#1}\checknextarg}
\newcommand{\checknextarg}{\@ifnextchar\bgroup{\gobblenextarg}{}}
\newcommand{\gobblenextarg}[1]{\,\mathrm{#1}\@ifnextchar\bgroup{\gobblenextarg}{}}
\makeatother

\newcommand{\avg}[1]{\left\langle #1 \right\rangle}
\newcommand{\angstrom}{\mbox{\normalfont\AA}}
\allowdisplaybreaks

\begin{document}

\tomtitle{}

\question{1a. \textbf{Hydrogen density}}
\answer{
    We can first calculate the density of the disc as
    \begin{align}
        \rho &= \frac{M}{\pi R_{\rm disc}^2 H} \\
             &= 1.9 \times 10^{-24} \unit{g}{cm^{-3}}
    \end{align}
    Now we need to convert the density to a number density as follows
    \begin{align}
        \rho &= m_{\rm H} n_{\rm H} + m_{\rm He} n_{\rm He} \\
             &= n_{\rm H} \qty(m_{\rm H} + m_{\rm He} \frac{n_{\rm He}}{n_{\rm H}}) \\
             &= n_{\rm H} \qty(m_{\rm H} + 0.1 m_{\rm He}) \\
             &= 1.4 n_{\rm H} m_p \\
        n_{\rm H} &= \frac{\rho}{1.4 m_p}
    \end{align}
    This gives that the average number density of hydrogen is
    \begin{equation}
        \boxed{ n_{\rm H} = 0.8 \unit{cm^{-3}} }
    \end{equation}
}

\question{1b. \textbf{Dust grain density}}
\answer{
    The mass of each dust grain is given by
    \begin{align}
        m_{\rm DG} &= \frac{4}{3} \pi R_{\rm RG}^3 \cdot \rho_{\rm RG} \\
                   &= 8.4 \times 10^{-15} \unit{g}
    \end{align}
    We are given the total dust mass and we already calculated the volume of the disc in part a so now the number density is just
    \begin{align}
        n_{\rm DG} &= \frac{M_{\rm dust}}{m_{\rm DG}} \cdot \frac{1}{V_{\rm disc}} \\
        \Aboxed{ n_{\rm DG} &= 1.6 \times 10^{-12} \unit{cm^{-3}} }
    \end{align}
}

\question{1c. \textbf{Molecular clouds - frequency and mass}}
\answer{
    First we can find the density of a cloud using the number density that we've been given and the conversion that we found in part a to account for the helium fraction (I'm ignoring dust here because it is such a small fraction).
    \begin{equation}
        \rho_{\rm MC} = 1.4 \cdot 2 m_p \cdot n(H_2)
    \end{equation}
    The typical mass can be found by multiplying the density and volume.
    \begin{align}
        M_{\rm MC} &= \rho_{\rm MC} V_{\rm MC} \\
                   &= 1.4 \cdot (2 m_{p}) \cdot n_{H_2} \cdot \qty(\frac{4}{3} \pi R_{\rm MC}^3) \\
        \Aboxed{ M_{\rm MC} &= 9.8 \times 10^{4} \unit{M_\odot} }
    \end{align}
    The approximate number of clouds in the galaxy is then just
    \begin{align}
        N_{\rm MC} &= \frac{0.3 M_{\rm disc}}{M_{\rm MC}} \\
        \Aboxed{ N_{\rm MC} &= 1.2 \times 10^{4} }
    \end{align}
}

\question{1d. \textbf{Molecular clouds getting in the way}}
\answer{
    If the centre of a molecular cloud passes within a radius of the line of sight then it will obscure our view. This defines the volume within which a molecular cloud could block our sight as
    \begin{equation}
        V_{\rm LoS} = \pi R_{\rm MC}^2 \cdot 8.5 \unit{kpc} = 6 \times 10^{6} \unit{pc^3}
    \end{equation}
    Given the total volume of the disc and the approximate number of molecular clouds in the disc, this implies that the expectation value of the number of clouds intersecting our line of sight is
    \begin{equation}
        \boxed{ \mathbb{E}[N_{\rm LoS}] = \frac{V_{\rm LoS}}{V_{\rm disc}} N_{\rm MC} = 0.52}
    \end{equation}
}

\end{document}

 