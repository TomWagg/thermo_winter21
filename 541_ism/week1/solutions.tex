\documentclass[12pt, letterpaper, twoside]{article}
\usepackage{nopageno,epsfig, amsmath, amssymb}
\usepackage{physics}
\usepackage{mathtools}
\usepackage{hyperref}
\usepackage{xcolor}
\hypersetup{
    colorlinks,
    linkcolor={blue},
    citecolor={blue},
    urlcolor={blue}
}
\usepackage{empheq}

\usepackage[letterpaper,
            margin=0.8in]{geometry}

\title{ASTR 541; Week 1}
\author{\textbf{Tom Wagg}}

\newcommand{\question}[1]{{\noindent \it #1}}
\newcommand{\answer}[1]{
    \par\noindent\rule{\textwidth}{0.4pt}#1\vspace{0.5cm}
}
\newcommand{\todo}[1]{{\color{red}\begin{center}TODO: #1\end{center}}}

% custom function for adding units
\makeatletter
\newcommand{\unit}[1]{%
    \,\mathrm{#1}\checknextarg}
\newcommand{\checknextarg}{\@ifnextchar\bgroup{\gobblenextarg}{}}
\newcommand{\gobblenextarg}[1]{\,\mathrm{#1}\@ifnextchar\bgroup{\gobblenextarg}{}}
\makeatother

\newcommand{\avg}[1]{\left\langle #1 \right\rangle}
\newcommand{\angstrom}{\mbox{\normalfont\AA}}
\allowdisplaybreaks

\begin{document}

\maketitle

\question{1a. \textbf{Hydrogen density}}
\answer{
    We can first calculate the density of the disc as
    \begin{align}
        \rho &= \frac{M}{\pi R_{\rm disc}^2 H} \\
             &= 1.9 \times 10^{-24} \unit{g}{cm^{-3}}
    \end{align}
    Now we need to convert the density to a number density as follows
    \begin{align}
        \rho &= m_{\rm H} n_{\rm H} + m_{\rm He} n_{\rm He} \\
             &= n_{\rm H} \qty(m_{\rm H} + m_{\rm He} \frac{n_{\rm He}}{n_{\rm H}}) \\
             &= n_{\rm H} \qty(m_{\rm H} + 0.1 m_{\rm He}) \\
             &= 1.4 n_{\rm H} m_p \\
        n_{\rm H} &= \frac{\rho}{1.4 m_p}
    \end{align}
    This gives that the average number density of hydrogen is
    \begin{equation}
        \boxed{ n_{\rm H} = 0.8 \unit{cm^{-3}} }
    \end{equation}
}



\end{document}

 