\documentclass[12pt, letterpaper, twoside]{article}
\usepackage{nopageno,epsfig, amsmath, amssymb}
\usepackage{physics}
\usepackage{mathtools}
\usepackage{hyperref}
\usepackage{xcolor}
\hypersetup{
    colorlinks,
    linkcolor={blue},
    citecolor={blue},
    urlcolor={blue}
}
\usepackage{empheq}

\usepackage[letterpaper,
            margin=0.8in]{geometry}

\title{ASTR 541; Week 1}
\author{\textbf{Tom Wagg}}

\newcommand{\question}[1]{{\noindent \it #1}}
\newcommand{\answer}[1]{
    \par\noindent\rule{\textwidth}{0.4pt}#1\vspace{0.5cm}
}
\newcommand{\todo}[1]{{\color{red}\begin{center}TODO: #1\end{center}}}

% custom function for adding units
\makeatletter
\newcommand{\unit}[1]{%
    \,\mathrm{#1}\checknextarg}
\newcommand{\checknextarg}{\@ifnextchar\bgroup{\gobblenextarg}{}}
\newcommand{\gobblenextarg}[1]{\,\mathrm{#1}\@ifnextchar\bgroup{\gobblenextarg}{}}
\makeatother

\newcommand{\avg}[1]{\left\langle #1 \right\rangle}
\newcommand{\angstrom}{\mbox{\normalfont\AA}}
\allowdisplaybreaks

\begin{document}

\maketitle

\question{1a. \textbf{Hydrogen density}}
\answer{
    We can first calculate the density of the disc as
    \begin{align}
        \rho &= \frac{M}{\pi R_{\rm disc}^2 H} \\
             &= 1.9 \times 10^{-24} \unit{g}{cm^{-3}}
    \end{align}
    Now we need to convert the density to a number density as follows
    \begin{align}
        \rho &= m_{\rm H} n_{\rm H} + m_{\rm He} n_{\rm He} \\
             &= n_{\rm H} \qty(m_{\rm H} + m_{\rm He} \frac{n_{\rm He}}{n_{\rm H}}) \\
             &= n_{\rm H} \qty(m_{\rm H} + 0.1 m_{\rm He}) \\
             &= 1.4 n_{\rm H} m_p \\
        n_{\rm H} &= \frac{\rho}{1.4 m_p}
    \end{align}
    This gives that the average number density of hydrogen is
    \begin{equation}
        \boxed{ n_{\rm H} = 0.8 \unit{cm^{-3}} }
    \end{equation}
}

\question{1b. \textbf{Dust grain density}}
\answer{
    ???
}

\question{1c. \textbf{Molecular clouds - frequency and mass}}
\answer{
    The typical mass can be found by multiplying the number density, mass of particles and volume.
    \begin{align}
        M_{\rm MC} &= m_{\rm H_2} n_{H_2} V_{\rm MC} \\
                   &= 2 m_{p} n_{H_2} \qty(\frac{4}{3} \pi R_{\rm MC}^3) \\
        \Aboxed{ M_{\rm MC} &= 10^{5} \unit{M_\odot} }
    \end{align}
    The approximate number of clouds in the galaxy is then just
    \begin{align}
        N_{\rm MC} &= \frac{0.3 M_{\rm disc}}{M_{\rm MC}} \\
        \Aboxed{ N_{\rm MC} &= 1.1 \times 10^{4} }
    \end{align}
}

\question{1d. \textbf{Molecular clouds getting in the way}}
\answer{
    If the centre of a molecular cloud passes within a radius of the line of sight then it will obscure our view. This defines the volume within which a molecular cloud could block our sight as
    \begin{equation}
        V_{\rm LoS} = \pi R_{\rm MC}^2 \cdot 8.5 \unit{kpc}
    \end{equation}
    Given the total volume of the disc and the approximate number of molecular clouds in the disc, this implies that the expectation value of the number of clouds intersecting our line of sight is
    \begin{equation}
        \boxed{ N = \frac{V_{\rm LoS}}{V_{\rm disc}} N_{\rm MC} = 0.73 }
    \end{equation}
}

\end{document}

 