\documentclass[12pt, letterpaper, twoside]{article}
\usepackage{nopageno,epsfig, amsmath, amssymb}
\usepackage{physics}
\usepackage{mathtools}
\usepackage{hyperref}
\usepackage{xcolor}
\hypersetup{
    colorlinks,
    linkcolor={blue},
    citecolor={blue},
    urlcolor={blue}
}
\usepackage{empheq}
\usepackage{wrapfig}

\usepackage[letterpaper,
            margin=0.8in]{geometry}

\newcommand{\psetnum}{0}
\newcommand{\class}{ASTR 541 - Interstellar Medium}

\newcommand{\tomtitle}{
    \noindent {\LARGE \fontfamily{cmr}\selectfont \textbf{\class}} \hfill \\[1\baselineskip]
    \noindent {\large \fontfamily{cmr}\selectfont Problem Set \psetnum \hfill \textsc{Tom Wagg}}\\[0.5\baselineskip]
    {\fontfamily{cmr}\selectfont \textit{\today}}\\[2\baselineskip]
}

\title{\class : Week \psetnum}
\author{\textbf{Tom Wagg}}

\newcommand{\question}[1]{{\noindent \it #1}}
\newcommand{\answer}[1]{
    \par\noindent\rule{\textwidth}{0.4pt}#1\vspace{0.5cm}
}
\newcommand{\todo}[1]{{\color{red}\begin{center}TODO: #1\end{center}}}

% custom function for adding units
\makeatletter
\newcommand{\unit}[1]{%
    \,\mathrm{#1}\checknextarg}
\newcommand{\checknextarg}{\@ifnextchar\bgroup{\gobblenextarg}{}}
\newcommand{\gobblenextarg}[1]{\,\mathrm{#1}\@ifnextchar\bgroup{\gobblenextarg}{}}
\makeatother

\newcommand{\avg}[1]{\left\langle #1 \right\rangle}
\newcommand{\angstrom}{\mbox{\normalfont\AA}}
\allowdisplaybreaks

\begin{document}

\tomtitle{}

\noindent For reference, if you'd ever like to see the code that I've used to get my answers to these, \href{https://github.com/TomWagg/uw-grad-classes/tree/main/541_ism}{here's a link to my GitHub repo}! (\#astropy.units for life)\\

\question{\textbf{1. Introduction to (Baryonic) Accounting}}

\noindent\emph{The general aim of this problem is to prove that the majority of baryonic matter does not lie in stars in galaxy or galaxies in clusters in a back-of-the-envelope calculation}
\answer{
    This answer heavily follows the solution laid out by \href{https://arxiv.org/pdf/astro-ph/9712020.pdf}{the paper that is the linked} in the question. To begin, we are given that $\Omega_B = 0.0455$ at $z = 0$. We can multiply this by the critical density,
    \begin{equation}
        \rho_c = \frac{3 H_0^2}{8 \pi G} \approx 9 \times 10^{-27} \unit{kg}{m^{-3}},
    \end{equation}
    to find that the mass density of baryonic matter is
    \begin{equation}
        \rho_B \approx 4 \times 10^{-28} \unit{kg}{m^{-3}}.
    \end{equation}

    \noindent Next we can calculate the mass density of stars and galaxies and compare it to this value. We can consider that most luminosity is coming from either a spheroid or a disk component and write that the density for each is
    \begin{equation}
        \rho = \mathcal{L}_B f_B \avg{M / L_B},
    \end{equation}
    where $\mathcal{L}_B$ is the mean luminosity density, $f_B$ is the fraction of light produced by this component and $\avg{M / L_B}$ is the mean mass-to-light ratio for this component.

    Let's assume that the only two components are spheroids and discs. The paper gives us that the fractions and mass-to-light ratios (in $\unit{M_{\odot}} / \unit{L_\odot}$) for each of these are $(0.385, 6.5)$ and $(0.582, 1.5)$ respectively. Summing these gives that the total mass density is
    \begin{equation}
        \rho = \mathcal{L}_B (0.385 \cdot 6.5 + 0.582 \cdot 1.5) \unit{M_{\odot}} / \unit{L_\odot} \approx 3.4 \mathcal{L}_B \unit{M_{\odot}} / \unit{L_\odot},
    \end{equation}
    From the paper we have that the mean luminosity density is approximately (assuming $h \approx 1$)
    \begin{equation}
        \mathcal{L}_B \approx 2 \times 10^{8} \unit{L_\odot}{Mpc^{-3}}
    \end{equation}
    Plugging this back in we get that the mass density of stars is
    \begin{equation}
        \rho_{\rm stars} \approx 6.8 \times 10^8 \unit{M_\odot}{Mpc^{-3}}
    \end{equation}\\

    \noindent We can now compare $\rho_B$ and $\rho_{\rm stars}$ and note that
    \begin{equation}
        \boxed{ \frac{\rho_{\rm stars}}{\rho_B} \approx 11\% }
    \end{equation}
    So from our back-of-the-envelope estimate we find that stars represent a small fraction of the baryonic matter within the Universe!!
}

\end{document}

 