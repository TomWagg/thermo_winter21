\documentclass[twocolumn]{aastex631}
\received{\today}
\shorttitle{Annotated Bibliography}
\graphicspath{{figures/}}

\usepackage{lipsum}
\usepackage{physics}
\usepackage{multirow}
\usepackage{xspace}
\usepackage{natbib}
\usepackage{fontawesome5}
\usepackage{xcolor}
\usepackage{wrapfig}
\usepackage[figuresright]{rotating}

% remove indents in footnotes
\usepackage[hang,flushmargin]{footmisc} 

\newcommand{\todo}[1]{{\color{red}{[TODO: #1}]}}
\newcommand{\needcite}{{\color{magenta}{(needs citation)}}}
\newcommand{\placeholder}[1]{{\color{gray} \lipsum[#1]}}

% custom function for adding units
\makeatletter
\newcommand{\unit}[1]{%
    \,\mathrm{#1}\checknextarg}
\newcommand{\checknextarg}{\@ifnextchar\bgroup{\gobblenextarg}{}}
\newcommand{\gobblenextarg}[1]{\,\mathrm{#1}\@ifnextchar\bgroup{\gobblenextarg}{}}
\makeatother

\begin{document}

\title{{\Large GOTO and Gravitational Wave Transient Followup}\\\vspace{0.15cm}ASTR 581 Annotated Bibliography}

% affiliations
\newcommand{\UW}{Department of Astronomy, University of Washington, Seattle, WA, 98195}

\author[0000-0001-6147-5761]{Tom Wagg}
\affiliation{\UW}

\correspondingauthor{Tom Wagg}
\email{tomwagg@uw.edu}

\section{Introduction}
For my bibliography I decided to investigate more about GOTO and transient follow up for gravitational wave events. I've previously worked on projects about LISA and LIGO predictions and heard a lot about GOTO so wanted to learn more.

GOTO, the gravitational wave optical transient observer, is a telescope designed specifically for quickly and efficiently following up on short-lived transients that occur after gravitational wave events. This sort of follow up is essential for connecting gravitational wave events to their electromagnetic counterparts and can help us learn more about the endpoints of massive stars, the neutron star equation of state and r-process enrichment.

Each of the papers that I discuss below consider the working of GOTO, design choices that went into it and some examples of how it can be used. I also touch on how other telescopes can do the same, in particular I focus on LSST.

\section{GOTO Telescope Control System}
\citet{Dyer+2018,Dyer+2020} describe the GOTO telescope control system and highlight how it can operate entirely robotically, with no need for human technicians except in the case of errors.

\section{GOTO Telescope}
\subsection{Prototype performance}
\citet{Steeghs+2022}
\placeholder{3}

\subsection{}
\citet{GOTO+2020}
\placeholder{4}

\section{GOTO in Practice}
\subsection{General}
\citet{Gompertz+2020}
\placeholder{5}

\subsection{Lightcurves}
\citet{Burhanudin+2021}
\placeholder{6}

\subsection{Kilonovae}
\citet{Chase+2022}
\placeholder{7}

\section{Other telescopes}
\subsection{LSST}
\citet{Andreoni+2022}
\placeholder{8}

\bibliographystyle{aasjournal}
\bibliography{bib}{}

\end{document}