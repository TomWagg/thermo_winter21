\documentclass[twocolumn]{aastex631}
\received{\today}
\shorttitle{Annotated Bibliography}
\graphicspath{{figures/}}

\usepackage{lipsum}
\usepackage{physics}
\usepackage{multirow}
\usepackage{xspace}
\usepackage{natbib}
\usepackage{fontawesome5}
\usepackage{xcolor}
\usepackage{wrapfig}
\usepackage[figuresright]{rotating}

% remove indents in footnotes
\usepackage[hang,flushmargin]{footmisc} 

\newcommand{\todo}[1]{{\color{red}{[TODO: #1}]}}
\newcommand{\needcite}{{\color{magenta}{(needs citation)}}}
\newcommand{\placeholder}[1]{{\color{gray} \lipsum[#1]}}

% custom function for adding units
\makeatletter
\newcommand{\unit}[1]{%
    \,\mathrm{#1}\checknextarg}
\newcommand{\checknextarg}{\@ifnextchar\bgroup{\gobblenextarg}{}}
\newcommand{\gobblenextarg}[1]{\,\mathrm{#1}\@ifnextchar\bgroup{\gobblenextarg}{}}
\makeatother

\begin{document}

\title{{\Large GOTO and Gravitational Wave Transient Followup}\\\vspace{0.15cm}ASTR 581 Annotated Bibliography}

% affiliations
\newcommand{\UW}{Department of Astronomy, University of Washington, Seattle, WA, 98195}

\author[0000-0001-6147-5761]{Tom Wagg}
% \affiliation{\UW}

\section{Introduction}
For my bibliography I decided to investigate more about GOTO and transient follow up for gravitational wave events. I've previously worked on projects about LISA and LIGO predictions and heard a lot about GOTO so wanted to learn more.

GOTO, the gravitational wave optical transient observer, is a telescope designed specifically for quickly and efficiently following up on short-lived transients that occur after gravitational wave events. This sort of follow up is essential for connecting gravitational wave events to their electromagnetic counterparts and can help us learn more about the endpoints of massive stars, the neutron star equation of state and r-process enrichment.

Each of the papers that I discuss below consider the working of GOTO, design choices that went into it and some examples of how it can be used. I also touch on how other telescopes can do the same, in particular I focus on LSST. Try and spot the moment where I realise I am running out of time to submit this and need to reduce the details (:

\section{GOTO Telescope Control System}
\citet{Dyer+2018,Dyer+2020} describe the GOTO telescope control system and highlight how it can operate entirely robotically, with no need for human technicians except in the case of errors.

The first of these papers \citep{Dyer+2018} outlines how the telescope control and scheduling system will work. GOTO is controlled by a series of daemons, the most important of which are the conditions, sentinel, scheduler and pilot daemons. The conditions daemon keeps track of the weather (e.g. rain, humidity, temperature, wind) and other system conditions and, if necessary, causes the dome to close or sets off an error. The sentinel daemon listens for alerts of new transients and adds them to database of potential pointings.

The scheduler then takes this database and ranks each of the potential transients based on a \textit{variety} of parameters, but in particular prioritises gravitational wave events over all other transients. If at any point there is no transient then GOTO performs an all-sky survey in order to perform image differencing at a later time. Importantly, the GOTO scheduler operates on a ``just-in-time'' scheduling model rather than creating a plan at the start of the night. This allows it to be very reactive to new events, though does mean that its choice of targets may be less efficient. Finally, the pilot daemon reacts to the input of the other daemons and skews the telescope, operates the dome, performs calibrations and takes images. If the pilot runs into any errors that can't be solved then it will alert a human technician but otherwise it is able to operate without any human intervention. The main concern stated in the paper is that the sky coverage may be decreased by the need for significant downtime in order to repeatedly exorcise a telescope that is run entirely by daemons \citep{2004exorcismis}. In summary, very cool!

\citet{Dyer+2020} follows up to this initial paper to further specify more clearly how alert scheduling will work and detail their ranking system \citep[e.g.\ see Eq.~1][]{Dyer+2020}. They now clarify how they will separate events into 3 sub-classes: gamma-ray bursts, gravitational wave events and gravitational wave event retractions (oop). The scheduler now also takes into account whether the gravitational wave event included a neutron star or included a black hole and the distance to the source. Closer sources with more neutron stars are prioritised over far sources with black holes (since closer sources are easier to observe and transients are not expected to as frequent for for black hole neutron star binaries and even less for binary black holes). Finally, this paper covers interesting possibilities of having multiple mounts (since the initial prototype). The two separate mounts can cover separate areas of the sky for greater coverage or they could focus on the same event and achieve greater depth. Decisions on exactly how they will put the multiple mounts is put off for the GOTO collaboration to decide in the future.

\section{GOTO Telescope}
Now let's talk more about the GOTO telescope in detail, which is described in \citet{GOTO+2020}. GOTO is built with speed in mind, additionally trying to maintain a wide field of view and depth. Each telescope is built around a fast-slewing German equatorial mount, which holds up to 8 unit telescopes. This modular nature is by design and allows a cost-effective large field of view whilst also allowing for more unit telescopes as more funding becomes available. They use the OnSemi KAF-50100 CCD, each telescope has a set of Baader LRGBC filters and the mount is housed in an 18ft Astrohaven clamshell dome.

The current plan is to have two independent mounts in separate domes in each site. Each mount will have 8 unit telescopes and thus provide a total instantaneous field of view of around 80 square degrees. They intend to have two sites, one in each hemisphere and thus attain all-sky coverage. However as of the writing of this paper they only have one site at the Observatorio del Roque de los Muchachos on La Palma in the Canary Islands. The second site is planned for Siding Spring Observatory in New South Wales, Australia and funding has been secured. In total this will give GOTO a field of view of 160 square degrees, which can observe to $\sim$ 20 mag in a set of three 60\,s exposures. As development continues, GOTO will be referred to as GOTO-X where X is the number of unit telescopes currently in operation.

\subsection{Prototype performance}
\citet{Steeghs+2022} goes into more detail about the GOTO prototype and how well it performed. I was interested to read that it was inspired by SuperWASP, I worked on this when I was 15 and did some research over the summer at Keele University. They also discuss how high-end professional large format CCDs would completely dominate costs and that they instead chose to use the more affordable range of Kodak sensors and make up for the lower quantum efficiency with a larger array of telescopes (and as a bonus they don't need fancy cooling systems).

This paper assesses the performance of the prototype in both the `triggered' and `sky-survey' mode. Recall these refer to when the telescope is looking at a specific target or performing an all-sky survey in between targets respectively. The prototype was deployed towards the end of LIGO-Virgo O2 with a view to being prepared for O3 and beyond. They found that the PSF performance (2.5 - 3.0 arcsec) was somewhat worse than expected (1.8 - 2.5 arcsec). Fortunately the prototype was still able to deliver the necessary sensitivity and depth \citep[see][Fig.~9]{Steeghs+2022}. Additionally, of the 8 gravitational-wave events that occurred on clear nights in the commissioning period, observations of all but one began with a mere 60 seconds and performance for GRB events was similar. In summary, the prototype went very well despite its worse than expected PSF performance.

\section{GOTO in Practice}
Now let's talk about how well GOTO has been performing in practice. \citet{Gompertz+2020} examine GOTO-4's performance during the LIGO-Virgo O3a operating run. They test the coverage of sources along the radial distance of the LIGO-Virgo collaboration (LVC) skymap using test sources: a kilonova-like function, a GRB along the jet axis and a GRB viewed off-axis. Using these test sources they analyse how well they followed up the 29 events from the O3a run. On average they began observations 8.79 hours after the preliminary alert from LVC. However, for observations that were unconstrained by the time of the alert (e.g. whether it was night or above the horizon), GOTO-4 observations began within a minute of the alert.

They find that on average, approximately 70.3\% of the available probability per event was tiled by GOTO-4. The probability here refers to the skymap that LVC provides, indicating a region in which the gravitational-wave event \textit{could} have occurred. So GOTO-4 is able to observe the majority of the sky in which an event could have occurred. However, an important distinction is that it doesn't necessarily have the depth to be able to cover the full \textit{volume} of the event. The percentage of volume covered actually appears to average below 1\%. One should note that GOTO mainly focusses on nearby binary neutron stars so this isn't as bad as it sounds.

The paper concludes that the main weakness of GOTO is currently its depth. As GOTO-4 becomes GOTO-16 and GOTO-32 this depth will increase as multiple mounts can image the same area and so they expect this to improve over time.

\subsection{Data Analysis with RNNs}
One of the intimidating factors of GOTO is the abundance of data that will be produced on a nightly basis. \citet{Burhanudin+2021} present a recurrent neural network (RNN) classifier that takes in photometric time-series data and additional contextual information (such as distance to nearby galaxies and on-sky position) to produce real-time classification of objects observed by GOTO. This RNN is extremely effective, achieving an AUC score of 0.972, and can even classify incomplete light curves that come, for example, from changing weather conditions. Accurate classification of objects is essential when dealing with the huge probability maps from LVC as one does not want to link an unrelated electromagnetic event to a gravitational wave event.

I do note that this paper only covers variable stars, SNe and AGNs but doesn't seem to characterise kilonovae which are (thought to be) one of the most common GW counterparts so I'd be intrigued to see how well it performs on that.

\subsection{Kilonovae}
And on the note of kilonovae, let's talk a bit more about them. \citet{Chase+2022} considers how well GOTO and other telescopes could see kilonovae. First they give a bit of background which I think is useful to share. Heavy lanthanides and actinides fuse in material gravitationally unbound during the coalescence of either binary neutron stars or neutron star black hole binaries with near-equal mass ratios. The residual radioactive decay of these r-process elements spurs electromagnetic emission, which is know as a kilonova. These events have typical timescales of about a day for the shorter wavelength emission and up to a week for the longer wavelength emission. To date, we've only detected one kilonova associated with a GW event: GW170818. This was mostly down to a series of fortunate events in terms of the sky localisation, distance and weather conditions at the telescope.

\citet{Chase+2022} show that the maximum redshift that one can detect 5/50/95\% of kilonovae is 0.097/0.029/0.0097 respectively. This is comparable to many other wide-field instruments, though it is not quite as good as LSST, but is superior to MeerLICHT, Swift and ZTF for instance. Of course much of the advantage of GOTO is that it is dedicated to looking for these events and can move to them quickly.
 
\section{LSST's capability for identifying counterparts}
Finally, as we mention other wide-field instruments, why don't we have a gander at everyone's favourite new shiny telescope. \citet{Andreoni+2022} considers how LSST could contribute to the follow-up of gravitational-wave events with target-of-opportunity observations. They show that LVK is expected to detect between 60-500 binary neutron stars during O5 (which is expected to last a year) and of those between 4-42 will have a localisation uncertainty below 20 square degrees. In addition to binary neutron stars and the expected kilonovae, Rubin will be able to put deep limits on the optical emission on binary black hole merger which is still quite uncertain.

\citet{Andreoni+2022} outline an observing strategy for dealing with various GW events in order to better characterise their counterparts. They propose observing around 7 well constrained NSNSs, 12 well constrained BHNSs and 2 BHBHs. In total this will require around 100 hours of LSST time over the course of a year (1.99\% of total time). Using this strategy, they expect an EM counterpart discovery in the vast majority of NSNS mergers within a distance of $\le$300 Mpc, assuming that GW170817 is not too dissimilar from the typical kilonovae from NSNS mergers. They will quantify their success in this first year and use to adjust the observing strategy in future years.

\bibliographystyle{aasjournal}
\bibliography{bib}{}

\end{document}