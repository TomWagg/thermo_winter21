\documentclass[twocolumn]{aastex631}
\received{\today}
\shorttitle{Annotated Bibliography}
\graphicspath{{figures/}}

\usepackage{lipsum}
\usepackage{physics}
\usepackage{multirow}
\usepackage{xspace}
\usepackage{natbib}
\usepackage{fontawesome5}
\usepackage{xcolor}
\usepackage{wrapfig}
\usepackage[figuresright]{rotating}

% remove indents in footnotes
\usepackage[hang,flushmargin]{footmisc} 

\newcommand{\todo}[1]{{\color{red}{[TODO: #1}]}}
\newcommand{\needcite}{{\color{magenta}{(needs citation)}}}
\newcommand{\placeholder}[1]{{\color{gray} \lipsum[#1]}}

% custom function for adding units
\makeatletter
\newcommand{\unit}[1]{%
    \,\mathrm{#1}\checknextarg}
\newcommand{\checknextarg}{\@ifnextchar\bgroup{\gobblenextarg}{}}
\newcommand{\gobblenextarg}[1]{\,\mathrm{#1}\@ifnextchar\bgroup{\gobblenextarg}{}}
\makeatother

\begin{document}

\title{{\Large GOTO and Gravitational Wave Transient Followup}\\\vspace{0.15cm}ASTR 581 Annotated Bibliography}

% affiliations
\newcommand{\UW}{Department of Astronomy, University of Washington, Seattle, WA, 98195}

\author[0000-0001-6147-5761]{Tom Wagg}
\affiliation{\UW}

\correspondingauthor{Tom Wagg}
\email{tomwagg@uw.edu}

\section{Introduction}
For my bibliography I decided to investigate more about GOTO and transient follow up for gravitational wave events. I've previously worked on projects about LISA and LIGO predictions and heard a lot about GOTO so wanted to learn more.

GOTO, the gravitational wave optical transient observer, is a telescope designed specifically for quickly and efficiently following up on short-lived transients that occur after gravitational wave events. This sort of follow up is essential for connecting gravitational wave events to their electromagnetic counterparts and can help us learn more about the endpoints of massive stars, the neutron star equation of state and r-process enrichment.

Each of the papers that I discuss below consider the working of GOTO, design choices that went into it and some examples of how it can be used. I also touch on how other telescopes can do the same, in particular I focus on LSST. Try and spot the moment where I realise I am running out of time to submit this and need to reduce the details (:

\section{GOTO Telescope Control System}
\citet{Dyer+2018,Dyer+2020} describe the GOTO telescope control system and highlight how it can operate entirely robotically, with no need for human technicians except in the case of errors.

The first of these papers \citep{Dyer+2018} outlines how the telescope control and scheduling system will work. GOTO is controlled by a series of daemons, the most important of which are the conditions, sentinel, scheduler and pilot daemons. The conditions daemon keeps track of the weather (e.g. rain, humidity, temperature, wind) and other system conditions and, if necessary, causes the dome to close or sets off an error. The sentinel daemon listens for alerts of new transients and adds them to database of potential pointings.

The scheduler then takes this database and ranks each of the potential transients based on a \textit{variety} of parameters, but in particular prioritises gravitational wave events over all other transients. If at any point there is no transient then GOTO performs an all-sky survey in order to perform image differencing at a later time. Importantly, the GOTO scheduler operates on a ``just-in-time'' scheduling model rather than creating a plan at the start of the night. This allows it to be very reactive to new events, though does mean that its choice of targets may be less efficient. Finally, the pilot daemon reacts to the input of the other daemons and skews the telescope, operates the dome, performs calibrations and takes images. If the pilot runs into any errors that can't be solved then it will alert a human technician but otherwise it is able to operate without any human intervention. Very cool!

\citet{Dyer+2020} follows up to this initial paper to further specify more clearly how alert scheduling will work and detail their ranking system \citep[e.g.\ see Eq.~1][]{Dyer+2020}. They now clarify how they will separate events into 3 sub-classes: gamma-ray bursts, gravitational wave events and gravitational wave event retractions (oop). The scheduler now also takes into account whether the gravitational wave event included a neutron star or included a black hole and the distance to the source. Closer sources with more neutron stars are prioritised over far sources with black holes (since closer sources are easier to observe and transients are not expected for binary black holes and are reduced for black hole neutron star binaries). Finally, this paper covers interesting possibilities of having multiple mounts (since the initial prototype). The two separate mounts can cover separate areas of the sky for greater coverage or they could focus on the same event and achieve greater depth. Decisions on exactly how they will put the multiple mounts is put off for the GOTO collaboration to decide in the future.

\section{GOTO Telescope}
Now let's talk more about the GOTO telescope in detail, which is described in \citet{GOTO+2020}. GOTO is built with speed in mind, additionally trying to maintain a wide field of view and depth. Each telescope is built around a fast-slewing German equatorial mount, which holds up to 8 unit telescopes. This modular nature is by design and allows a cost-effective large field of view whilst also allowing for more unit telescopes as more funding becomes available. They use the OnSemi KAF-50100 CCD, each telescope has a set of Baader LRGBC filters and the mount is housed in an 18ft Astrohaven clamshell dome.

The current plan is to have two independent mounts in separate domes in each site. Each mount will have 8 unit telescopes and thus provide a total instantaneous field of view of around 80 square degrees. They intend to have two sites, one in each hemisphere and thus attain all-sky coverage. However as of the writing of this paper they only have one site at the Observatorio del Roque de los Muchachos on La Palma in the Canary Islands. The second site is planned for Siding Spring Observatory in New South Wales, Australia and funding has been secured. In total this will give GOTO a field of view of 160 square degrees, which can observe to $\sim$ 20 mag in a set of three 60\,s exposures.

\subsection{Prototype performance}
\citet{Steeghs+2022} goes into more detail about the GOTO prototype and how well it performed. I was interested to read that it was inspired by SuperWASP, I worked on this when I was 15 and did some research over the summer at Keele University! They also discuss how high-end professional large format CCDs would completely dominate costs and that they instead chose to use the more affordable range of Kodak sensors and make up for the lower quantum efficiency with a larger array of telescopes (and as a bonus they don't need fancy cooling systems).

This paper assesses the performance of the prototype in both the `triggered' and `sky-survey' mode. Recall these refer to when the telescope is looking at a specific target or performing an all-sky survey in between targets respectively. The prototype was deployed towards the end of LIGO-Virgo O2 with a view to being prepared for O3 and beyond. They found that the PSF performance (2.5 - 3.0 arcsec) was somewhat worse than expected (1.8 - 2.5 arcsec). Fortunately the prototype was still able to deliver the necessary sensitivity and depth \citep[see][Fig.~9]{Steeghs+2022}. Additionally, of the 8 gravitational-wave events that occurred on clear nights in the commissioning period, observations of all but one began with a mere 60 seconds and performance for GRB events was similar. In summary, the prototype went very well despite its worse than expected PSF performance.

\section{GOTO in Practice}
\subsection{General}
\citet{Gompertz+2020}
\placeholder{5}

\subsection{Lightcurves}
\citet{Burhanudin+2021}
\placeholder{6}

\subsection{Kilonovae}
\citet{Chase+2022}
\placeholder{7}

\section{Other telescopes}
\subsection{LSST}
\citet{Andreoni+2022}
\placeholder{8}

\bibliographystyle{aasjournal}
\bibliography{bib}{}

\end{document}